\newlength\mystoreparindent
\newenvironment{myparindent}[1]{%
\setlength{\mystoreparindent}{\the\parindent}
\setlength{\parindent}{#1}
}{%
\setlength{\parindent}{\mystoreparindent}
}

\documentclass[a4paper]{article}
\usepackage{amssymb}
\usepackage{amsthm}
\usepackage{mathtools}
\usepackage{enumitem}
\usepackage{graphicx}
\usepackage{mathrsfs}
\usepackage{amsmath}

\DeclarePairedDelimiterX{\infdivx}[2]{(}{)}{%
  #1\;\delimsize\|\;#2%
}
\newcommand{\infdiv}{D\infdivx}
\DeclarePairedDelimiter{\norm}{\lVert}{\rVert}

\newtheorem{theorem}{Theorem}[section]
\newtheorem{corollary}{Corollary}[theorem]
\newtheorem{lemma}[theorem]{Lemma}

\graphicspath{ {./} }

\begin{document}

\begin{myparindent}{0pt}

Kyle Kloberdanz \newline
5 March 2022 \newline

\textbf{Exercise 3.3.1}:
Suppose $\phi: G \rightarrow H$ is a homomorphism, and $g \in G$. Prove that
for all $n > 0$ we have $\phi(g^n) = \phi(g)^n$ by induction on $n$, thus
completing the proof for Proposition 3.3.4
\newline

\begin{proof}

Base case, let $n = 1$

\[ \phi(g^1) = \phi(g) = \phi(g)^1 \]

Now, let n = 2

\[ \phi(g^2) = \phi(g^{1 + 1}) = \phi(gg) \]

Because $\phi$ is a homomorphism

\[ \phi(gg) = \phi(g) \phi(g) = \phi(g)^2 \]

For $n = 3$

\[ \phi(g^3) = \phi(g^{1 + 2}) = \phi(gg^2) = \phi(g) \phi(g^2) \]

But we have just seen that $\phi(g^2) = \phi(g)^2$, so

\[ \phi(g) \phi(g^2) = \phi(g) \phi(g)^2 = \phi(g) \phi(g) \phi(g) = \phi(g)^3 \]

By continuing this for any $n \in \mathbb{Z}$
\newline

$\phi(g^n) = \phi(g^{n - 1}g) = \phi(g^{n - 1}) \phi(g) = \phi(g^{n - 2}g) \phi(g) =$

$\phi(g^{n-2}) \phi(g) \phi(g) = \phi(g) ...$ n times $... \phi(g) = \phi(g)^n$
\newline

Therefore $\phi(g^n) = \phi(g)^n$
\end{proof}

\textbf{Exercise 3.3.4}:
Prove that if $G$ is an abelian group, then every subgroup of $G$ is normal.
\newline

\begin{proof}

Let $N < G$

Recall, a normal group is a group where for all $g \in G$, $gNg^{-1} = N$.

Because $N$ and $G$ are abelian, $gNg^{-1} = Ngg^{-1} = Ne = N$

Therefore, if $G$ is an abelian group, then every subgroup of $G$ is normal.
\end{proof}

\textbf{Exercise 3.3.6}:
Prove that for any subgroup $H < G$ and element $g \in G$, $gHg^{-1}$ is also
a subgroup of $G$, and that $c_g(h) = ghg^{-1}$ defines an isomorphism
$c_g: H \rightarrow gHg^{-1}$. In particular, if $H \triangleleft G$, then
conjugation in $G$ defines and automorphism $c_g: H \rightarrow H$.
\newline

\textbf{Exercise 3.3.8}:
Let $G$ be a group and $H < G$ a subgroup. Prove that the set

\[ N(H) = \{g \in G | gHg^{-1} = H \} \subset G \]
\newline

is a subgroup containing $H$, and that $H \triangleleft N(H)$.

\textit{The subgroup N(H) from Exercise 3.3.8 is called the
\textbf{normalizer of H}, and it is (by definition) the largest subgroup of G
containing H in which H is normal}.
\newline

\textbf{Exercise 3.3.10}:
Prove that if $gcd(n, m) = 1$, then
$\mathbb{Z}_{nm} \cong \mathbb{Z}_n \times \mathbb{Z}_m$
(recall that in $\mathbb{Z}_n^\times$, the operation is multiplication of
congruence classes). \textit{Hint: Theorem 1.5.8 and the discussion there}.
\newline

\textbf{Exercise 3.3.13}:
An \textbf{ideal} (or sometimes called a \textbf{two-sided ideal}) is a
subring $\mathcal{J} \subset R$ with the property that for all
$r \in R$ and $a \in \mathcal{J}$, we have $ar, ra \in \mathcal{J}$.
Prove that the kernel of a ring homomorphism $\phi: R \rightarrow S$ is an ideal.
\newline

\begin{proof}
  Recall, a kernel of $\phi$ is defined as:
  \[ ker(\phi) = \{ g \in G | \phi(g) = e \} \]

  Hence, the kernel of $\phi$ is:
  \[ ker(\phi) = \{ r \in R | \phi(r) = 0 \} \]

  Let $r \in R$ and $a \in ker(\phi)$. We can see that $\phi(a) = 0$

  \[ \phi(r \times a) = \phi(r) \times \phi(a) = \phi(r) \times 0 = 0 \]
  \[ \phi(a \times r) = \phi(a) \times \phi(r) = 0 \times \phi(r) = 0 \]

  Because $\mathcal{J}$ is a ring, $0 \in \mathcal{J}$, hence $ar, ra \in \mathcal{J}$,
  therefore $\phi: R \rightarrow S$ is an ideal.
\end{proof}

\textbf{Exercise 3.3.16}:
Let $\phi: \mathbb{Z} \rightarrow \mathbb{Z}$ be given by $\phi(k) = 2k$. Prove
that although $\phi$ is a homomorphism of additive groups, it is \textit{not}
a ring homomorphism.

\begin{proof}
  First, we will demonstrate that $\phi$ is indeed a group homomorphism. \newline

  Let $x, y \in \mathbb{Z}$ and let $k = a + b$
  \[ \phi(k) = \phi(a + b) = \phi(a) + \phi(b) = 2 \times a + 2 \times b =
  2(a + b) = 2k
  \]

  We can see that $\phi$ is indeed a group homomorphism. \newline

  Recall, a ring homomorphism requires that both addition and multiplication
  are preserved. \newline

  We will demonstrate a counter example in which addition is preserved, but
  multiplication is not. Given that integers form a group under addition,
  and as we have shown above, $\phi: \mathbb{Z} \rightarrow \mathbb{Z}$ is a
  group homomorphism, we know that addition will be preserved. \newline

  Let $k = 10$.
  \[ \phi(10) = 2 \times 10 = 20 \]
  \[ \phi(10) = \phi(2 + 8) = \phi(2) + \phi(8) = 2 \times 2 + 2 \times 8 = 4 + 16 = 20 \]

  So far so good. While this doesn't prove anything in itself, we have now seen an
  example of $\phi$ preserving addition. \newline

  Now let's try multiplication.
  \[ \phi(10) = 2 \times 10 = 20 \]

  Let's see what happens when we substitute $10$ for $2 \times 5$ and assume that
  $\phi$ will form a homomorphism under multiplication.

  \[ 10 = 2 \times 5 \]
  \[ \phi(2 \times 5) = \phi(2) \times \phi(5) = 2 \times 2 \times 2 \times 5 = 40 \]
  \[ 20 \neq 40 \]

  Because the homomorphism is not preserved under multiplication, this therefore
  demonstrates a counterexample that $\phi$ is not a ring homomorphism.

  A more general way to demonstrate this is as follows, assume that $\phi$
  forms a homomorphism under multiplication. \newline

  Let $x, y \in \mathbb{Z}$ and $k = x \times y$.

  \[ \phi(k) = \phi(x \times y) = \phi(x) \times \phi(y) = 2x \times 2y = 4(x \times y) =
  2 \times 2(x \times y) = 2 \times 2k \neq 2k \]

  We can see that while $\phi: \mathbb{Z} \rightarrow \mathbb{Z}$ forms a group
  homomorphism under addition, it does not preserve multiplication, and is
  therefore not a ring homomorphism.

\end{proof}

\end{myparindent}
\end{document}
