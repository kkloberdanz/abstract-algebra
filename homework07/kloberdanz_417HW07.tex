\newlength\mystoreparindent
\newenvironment{myparindent}[1]{%
\setlength{\mystoreparindent}{\the\parindent}
\setlength{\parindent}{#1}
}{%
\setlength{\parindent}{\mystoreparindent}
}

\documentclass[a4paper]{article}
\usepackage{amssymb}
\usepackage{amsthm}
\usepackage{mathtools}
\usepackage{enumitem}
\usepackage{graphicx}
\usepackage{mathrsfs}
\usepackage{amsmath}

\DeclarePairedDelimiterX{\infdivx}[2]{(}{)}{%
  #1\;\delimsize\|\;#2%
}
\newcommand{\infdiv}{D\infdivx}
\DeclarePairedDelimiter{\norm}{\lVert}{\rVert}

\newtheorem{theorem}{Theorem}[section]
\newtheorem{corollary}{Corollary}[theorem]
\newtheorem{lemma}[theorem]{Lemma}

\graphicspath{ {./} }

\begin{document}

\begin{myparindent}{0pt}

Kyle Kloberdanz \newline
5 March 2022 \newline

\textbf{Exercise 3.3.1}:
Suppose $\phi: G \rightarrow H$ is a homomorphism, and $g \in G$. Prove that
for all $n > 0$ we have $\phi(g^n) = \phi(g)^n$ by induction on $n$, thus
completing the proof for Proposition 3.3.4
\newline

\begin{proof}

Base case, let $n = 1$

\[ \phi(g^1) = \phi(g) = \phi(g)^1 \]

Now, let n = 2

\[ \phi(g^2) = \phi(g^{1 + 1}) = \phi(gg) \]

Because $\phi$ is a homomorphism

\[ \phi(gg) = \phi(g) \phi(g) = \phi(g)^2 \]

For $n = 3$

\[ \phi(g^3) = \phi(g^{1 + 2}) = \phi(gg^2) = \phi(g) \phi(g^2) \]

But we have just seen that $\phi(g^2) = \phi(g)^2$, so

\[ \phi(g) \phi(g^2) = \phi(g) \phi(g)^2 = \phi(g) \phi(g) \phi(g) = \phi(g)^3 \]

By continuing this for any $n \in \mathbb{Z}$
\newline

$\phi(g^n) = \phi(g^{n - 1}g) = \phi(g^{n - 1}) \phi(g) = \phi(g^{n - 2}g) \phi(g) =$

$\phi(g^{n-2}) \phi(g) \phi(g) = \phi(g) ...$ n times $... \phi(g) = \phi(g)^n$
\newline

Therefore $\phi(g^n) = \phi(g)^n$
\end{proof}

\textbf{Exercise 3.3.4}:
Prove that if $G$ is an abelian group, then every subgroup of $G$ is normal.
\newline

\begin{proof}

Let $N < G$

Recall, a normal group is a group where for all $g \in G$, $gNg^{-1} = N$.

Because $N$ and $G$ are abelian, $gNg^{-1} = Ngg^{-1} = Ne = N$

Therefore, if $G$ is an abelian group, then every subgroup of $G$ is normal.
\end{proof}

\textbf{Exercise 3.3.6}:
Prove that for any subgroup $H < G$ and element $g \in G$, $gHg^{-1}$ is also
a subgroup of $G$, and that $c_g(h) = ghg^{-1}$ defines an isomorphism
$c_g: H \rightarrow gHg^{-1}$. In particular, if $H \triangleleft G$, then
conjugation in $G$ defines and automorphism $c_g: H \rightarrow H$.
\newline

\textbf{Exercise 3.3.8}:
Let $G$ be a group and $H < G$ a subgroup. Prove that the set

\[ N(H) = \{g \in G | gHg^{-1} = H \} \subset G \]
\newline

is a subgroup containing $H$, and that $H \triangleleft N(H)$.

\textit{The subgroup N(H) from Exercise 3.3.8 is called the
\textbf{normalizer of H}, and it is (by definition) the largest subgroup of G
containing H in which H is normal}.
\newline

\textbf{Exercise 3.3.10}:
Prove that if $gcd(n, m) = 1$, then
$\mathbb{Z}_{nm} \cong \mathbb{Z}_n \times \mathbb{Z}_m$
(recall that in $\mathbb{Z}_n^\times$, the operation is multiplication of
congruence classes). \textit{Hint: Theorem 1.5.8 and the discussion there}.
\newline

\textbf{Exercise 3.3.13}:
An \textbf{ideal} (or sometimes called a \textbf{two-sided ideal}) is a
substring TODO
\newline

\textbf{Exercise 3.3.16}:
Let $\phi: \mathbb{Z} \rightarrow \mathbb{Z}$ be given by $\phi(k) = 2k$. Prove
that although $\phi$ is a homomorphism of additive groups, it is \textit{not}
a ring homomorphism.

\end{myparindent}
\end{document}
