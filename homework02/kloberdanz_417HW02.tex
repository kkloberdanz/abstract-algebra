\newlength\mystoreparindent
\newenvironment{myparindent}[1]{%
\setlength{\mystoreparindent}{\the\parindent}
\setlength{\parindent}{#1}
}{%
\setlength{\parindent}{\mystoreparindent}
}

\documentclass[a4paper]{article}
\usepackage{amssymb}
\usepackage{amsthm}
\usepackage{mathtools}

\DeclarePairedDelimiterX{\infdivx}[2]{(}{)}{%
  #1\;\delimsize\|\;#2%
}
\newcommand{\infdiv}{D\infdivx}
\DeclarePairedDelimiter{\norm}{\lVert}{\rVert}

\newtheorem{theorem}{Theorem}[section]
\newtheorem{corollary}{Corollary}[theorem]
\newtheorem{lemma}[theorem]{Lemma}

\begin{document}

\begin{myparindent}{0pt}

Kyle Kloberdanz \newline
17 January 2022 \newline

\textbf{Exercise 1.4.2}:
Prove that if $a, b, c, m, n \in \mathbb{Z}$, $a|c$, and $a|c$, then
$a|(mb + nc)$
\begin{proof}
Recall that $a|b$ implies $b = xa, x \in \mathbb{Z}$
and $a|c$ implies $c = ya, y \in \mathbb{Z}$. Hence we can write
$b + c = xa + ya$. We can see that $a | xa$ and $a | ya$, hence $a | (xa + ya)$,
therefore $a | (b + c)$.
\newline

It is also true that $a | xam$ and $a | yan$, hence $a | (xam + yan)$,
therefore \newline
$a | (mb + nc)$
\end{proof}

\textbf{Exercise 1.4.3}:
For each of the pairs $(a, b) = (130, 95), (1295, 406), (1351, 165)$, find $gcd(a, b)$
using the Euclidean Algorithm and express it in the form $gcd(a, b) = m_0 a + n_0 b$
for $m_0, n_0 \in \mathbb{Z}$. \newline

$gcd(130, 95)$ \newline
$130 = 95 \times q + r$ \newline
$130 = 95 \times 1 + 35$ \newline
$95 = 35 \times 2 + 25$ \newline
$35 = 25 \times 1 + 10$ \newline
$25 = 10 \times 2 + 5$ \newline
$10 = 5 \times 2 + 0$ \newline

Therefore, $gcd(130, 95) = 5$ \newline

$gcd(1295, 406)$ \newline
$1295 = 406 \times q + r$ \newline
$1295 = 406 \times 3 + 77$ \newline
$406 = 77 \times 5 + 21$ \newline
$77 = 21 \times 3 + 14$ \newline
$21 = 14 \times 1 + 7$ \newline
$14 = 7 \times 2 + 0$ \newline

Therefore, $gcd(1295, 406) = 7$ \newline

$gcd(1351, 165)$ \newline
$1351 = 165 \times q + r$ \newline
$1351 = 165 \times 8 + 31$ \newline
$165 = 31 \times 5 + 10$ \newline
$31 = 10 \times 3 + 1$ \newline
$10 = 3 \times 3 + 1$ \newline
$3 = 1 \times 3 + 0$ \newline

Therefore, $gcd(1351, 165) = 1$ \newline
\newline

\textbf{Exercise 1.4.4}:
Suppose $a, b, c \in \mathbb{Z}$. Prove that if $gcd(a, b) = 1, a|c, b|c,$ then $ab|c$
\begin{proof}
Recall that $gcd(a, b) = 1$ implies $ma + nb = 1$ where $m, n \in \mathbb{Z}$.
Also recall that $a | c$ implies $c = xa$ and $b | c$ implies $c  = yb$
where $x, y \in \mathbb{Z}$.

Given the facts above, we can multiply $ma + nb = 1$ by $c$ to find that
$c = cma + cnb$. Substituting for $c$, we find $c = ybma + xanb$. Factoring out
$ab$ yields $c = ab(ym + xn)$, therefore $ab | c$
\end{proof}

\textbf{Exercise 1.5.2}:
Write down the addition and multiplication tables for $\mathbb{Z}_5$ \newline

\begin{center}
\begin{tabular}{ c| c | c | c | c | c | c |}
$+$ & 0 & 1 & 2 & 3 & 4 \\
\hline
0 & 0 & 1 & 2 & 3 & 4 \\
\hline
1 & 1 & 2 & 3 & 4 & 0 \\
\hline
2 & 2 & 3 & 4 & 0 & 1 \\
\hline
3 & 3 & 4 & 0 & 1 & 2 \\
\hline
4 & 4 & 0 & 1 & 2 & 3 \\
\hline
\end{tabular}
\end{center}

\begin{center}
\begin{tabular}{ c| c | c | c | c | c | c |}
$\times$ & 0 & 1 & 2 & 3 & 4 \\
\hline
0 & 0 & 0 & 0 & 0 & 0 \\
\hline
1 & 0 & 1 & 2 & 3 & 4 \\
\hline
2 & 0 & 2 & 4 & 1 & 3 \\
\hline
3 & 0 & 3 & 1 & 4 & 2 \\
\hline
4 & 0 & 4 & 3 & 2 & 1 \\
\hline
\end{tabular}
\end{center}

\textbf{Exercise 1.5.3}:
List all elements of $\mathbb{Z}_5^\times, \mathbb{Z}_6^\times, \mathbb{Z}_8^\times$, and $\mathbb{Z}_{20}^\times$.
\newline

Recall Proposition 1.5.6. \textit{For all $n \ge 1$, we have} $\mathbb{Z}_n^\times = \{ [a] \in \mathbb{Z}_n | gcd(a, n) = 1 \}$ \newline

$\mathbb{Z}_5^\times = \{ 1, 2, 3, 4 \}$ \newline
$\mathbb{Z}_6^\times = \{ 1, 5 \}$ \newline
$\mathbb{Z}_8^\times = \{ 1, 3, 5, 7 \}$ \newline
$\mathbb{Z}_{20}^\times = \{ 1, 3, 7, 9, 11, 13, 17, 19 \}$ \newline

As an aside, this problem can be solved with a beautiful one-liner in Haskell. \
\begin{verbatim}
Prelude> map (\n -> [x | x <- [1..n], gcd x n == 1]) [5, 6, 8, 20]
[[1,2,3,4],[1,5],[1,3,5,7],[1,3,7,9,11,13,17,19]]
\end{verbatim}

\textbf{Exercise 1.5.4}:
Prove that if $m|n$, then $\pi_{m, n}: \mathbb{Z}_n \rightarrow \mathbb{Z}_m$ is well defined.

\begin{proof}

Recall from the text that $\pi_{m, n}([a]_n) = [a]_m$.
Recall that a function, $f$, is well-defined if $[x] = [y] \implies f(x) = f(y)$.
Recall Proposition 1.4.2 (iv), which states if $a|b$ and $b|c$, then $a|c$ \newline

Therefore, we must show that: \newline

If $m | n$ and $[x]_n = [y]_n$ where $x, y \in \mathbb{Z}$, then
$\pi_{m, n}(x) = \pi_{m, n}(y)$ where $\pi_{m, n}: \mathbb{Z}_n \rightarrow \mathbb{Z}_m$
and therefore $[x]_m = [y]_m$. \newline

$[x]_n = [y]_n$ implies $x \equiv y \pmod{n}$, hence $n | (x - y)$. \newline

Because $m | n$ and $n | (x - y)$, then by Proposition 1.4.2 (iv), we see that
$m | (x - y)$, hence $x \equiv y \pmod{m}$, hence $[x]_m = [y]_m$,
therefore $\pi_{m, n}$ is well-defined.

\end{proof}

\end{myparindent}
\end{document}
