\newlength\mystoreparindent
\newenvironment{myparindent}[1]{%
\setlength{\mystoreparindent}{\the\parindent}
\setlength{\parindent}{#1}
}{%
\setlength{\parindent}{\mystoreparindent}
}

\documentclass[a4paper]{article}
\usepackage{amssymb}
\usepackage{amsthm}
\usepackage{mathtools}

\DeclarePairedDelimiterX{\infdivx}[2]{(}{)}{%
  #1\;\delimsize\|\;#2%
}
\newcommand{\infdiv}{D\infdivx}
\DeclarePairedDelimiter{\norm}{\lVert}{\rVert}

\newtheorem{theorem}{Theorem}[section]
\newtheorem{corollary}{Corollary}[theorem]
\newtheorem{lemma}[theorem]{Lemma}

\begin{document}

\begin{myparindent}{0pt}

Kyle Kloberdanz \newline
12 December 2021 \newline

\textbf{Exercise 1.4.2}:
Prove that if $a, b, c, m, n \in \mathbb{Z}$, $a|c$, and $a|c$, then $a|(mb + nc)$
\begin{proof}
\end{proof}

\textbf{Exercise 1.4.3}:
For each of the pairs $(a, b) = (130, 95), (1295, 406), (1351, 165)$, find $gcd(a, b)$
using the Euclidean Algorithm and express it in the form $gcd(a, b) = m_0 a + n_0 b$
for $m_0, n_0 \in \mathbb{Z}$. \newline

$gcd(130, 95)$ \newline
$130 = 95 \times q + r$ \newline
$130 = 95 \times 1 + 35$ \newline
$95 = 35 \times 2 + 25$ \newline
$35 = 25 \times 1 + 10$ \newline
$25 = 10 \times 2 + 5$ \newline
$10 = 5 \times 2 + 0$ \newline

Therefore, $gcd(130, 95) = 5$ \newline

$gcd(1295, 406)$ \newline
$1295 = 406 \times q + r$ \newline
$1295 = 406 \times 3 + 77$ \newline
$406 = 77 \times 5 + 21$ \newline
$77 = 21 \times 3 + 14$ \newline
$21 = 14 \times 1 + 7$ \newline
$14 = 7 \times 2 + 0$ \newline

Therefore, $gcd(1295, 406) = 7$ \newline

$gcd(1351, 165)$ \newline
$1351 = 165 \times q + r$ \newline
$1351 = 165 \times 8 + 31$ \newline
$165 = 31 \times 5 + 10$ \newline
$31 = 10 \times 3 + 1$ \newline
$10 = 3 \times 3 + 1$ \newline
$3 = 1 \times 3 + 0$ \newline

Therefore, $gcd(1351, 165) = 1$ \newline
\newline

\textbf{Exercise 1.4.4}:
Suppose $a, b, c \in \mathbb{Z}$. Prove that if $gcd(a, b) = 1, a|c, b|c,$ then $ab|c$
\begin{proof}
\end{proof}

\textbf{Exercise 1.5.2}:
Write down the addition and multiplication tables for $\mathbb{Z}_5$

\textbf{Exercise 1.5.3}:
List all elements of $\mathbb{Z}_5^x, \mathbb{Z}_6^x, \mathbb{Z}_8^x$, and $\mathbb{Z}_{20}^x$.

\textbf{Exercise 1.5.4}:
Prove that if $m|n$, then $\pi_{m, n}: \mathbb{Z}_n \rightarrow \mathbb{Z}_m$ is well defined.

\begin{proof}
\end{proof}

\end{myparindent}
\end{document}
