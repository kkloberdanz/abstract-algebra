\newlength\mystoreparindent
\newenvironment{myparindent}[1]{%
\setlength{\mystoreparindent}{\the\parindent}
\setlength{\parindent}{#1}
}{%
\setlength{\parindent}{\mystoreparindent}
}

\documentclass[a4paper]{article}
\usepackage{amssymb}
\usepackage{amsthm}
\usepackage{mathtools}

\DeclarePairedDelimiterX{\infdivx}[2]{(}{)}{%
  #1\;\delimsize\|\;#2%
}
\newcommand{\infdiv}{D\infdivx}
\DeclarePairedDelimiter{\norm}{\lVert}{\rVert}

\newtheorem{theorem}{Theorem}[section]
\newtheorem{corollary}{Corollary}[theorem]
\newtheorem{lemma}[theorem]{Lemma}

\begin{document}

\begin{myparindent}{0pt}

Kyle Kloberdanz \newline
20 November 2021 \newline

\textbf{Exercise 1.1.1}:
Given functions $\sigma: A \rightarrow B$ and $\tau: B \rightarrow C$,
prove that if $\tau \circ \sigma$ is injective, then so is $\sigma$.

\begin{proof}
Let's setup a proof by contradiction, assume $\sigma$ is not injective, then by
definition there exists two distinct values $a_1$, $a_2$ $\in A$ such that
$\sigma(a_1) = \sigma(a_2)$.
This implies that $\tau(\sigma(a_1)) = \tau(\sigma(a_2))$
which contradicts the assumption that $x \in A$, $\tau(\sigma(x))$ is an injective function.
Hence $a_1$ and $a_2$ cannot exist and $\sigma$ must be an injective function.
\end{proof}

\textbf{Exercise 1.2.2}:
Decide which of the following are equivalence relations on the set of natural numbers $\mathbb{N}$. For those
that are, prove it. For those that are not, explain why. \newline
\newline

(i) $x \sim y$ if $\norm{x - y} \le 3$ \newline
\begin{lemma}
Before we begin our proof, let's first consider that for
$x, y \in \mathbb{N}, \norm{x - y} = \norm{y - x}$. Let's now prove this lemma.
\newline
\begin{proof}
$\norm{x - y} = \norm{x - y}$ \newline
$\norm{x - y} = \norm{1} \times \norm{x - y}$ \newline
$\norm{x - y} = \norm{-1} \times \norm{x - y}$ \newline
$\norm{x - y} = \norm{-1 \times (x - y)}$ \newline
$\norm{x - y} = \norm{-x + y}$ \newline
$\norm{x - y} = \norm{y - x}$ \newline
We have therefore shown that $x, y \in \mathbb{N}, \norm{x - y} = \norm{y - x}$
and we will refer to this lemma in a future proof.
\end{proof}
\end{lemma}

\begin{proof}
First, for any $x \in \mathbb{N}, x - x = 0$. Furthermore $\norm{x - x} = 0 \le 3$.
hense, $x \sim x$, and therefore we have demonstrated that this statement is reflexive. \newline
\newline
Second, given the above lemma proving that $\norm{x - y} = \norm{y - x}$, we can see that
$\norm{x - y} \le 3$ is the same as $\norm{y - x} \le 3$, hence $x \sim y$ and $y \sim x$
therefore we have demonstrated that this statement is symmetric. \newline
\newline
Third, if
\end{proof}

(ii) $x \sim y$ if $\norm{x - y} \ge 3$ \newline
\begin{proof}
For any $x \in \mathbb{N}, x - x = 0$. Furthermore $\norm{x - x} = 0$. Clearly 0 is less than 3,
hence the statement $\norm{x - x} \ge 3$ is false,
hence $x \sim x$ is false, which means $x \sim y$ is not reflexive and therefore not an equivalence relation. \newline
\end{proof}

(iii) $x \sim y$ if $x$ and $y$ have the same digit in the 1's place (expressed in base 10). \newline
\begin{proof}
Let the set $D$ be the set of decimal digits, which would be exactly $\{0, 1, 2, 3, 4, 5, 6, 7, 8, 9\}$.
Note that $D \subset \mathbb{N}$.
Let $x_1, y_1, z_1 \in D$ be the digit in the one's place of $x, y, z \in \mathbb{N}$ respectively. \newline
\newline

First, because for any natural number $n \in \mathbb{N}, n$ is the same as itself,
hence $x_1 = x_1$ and $y_1 = y_1$, and therefore $x \sim x$ and $y \sim y$ so it is reflexive. \newline
\newline

Second, observe that if $x_1 = y_1$, then $x \sim y$, furthermore $x_1$ is the same number as $y_1$, but then $y_1$ is the same number as $x_1$,
and hence $y \sim x$, so it is symmetric. \newline
\newline
Third, if $x_1 = y_1$ then $x \sim y$ and if $y_1 = z_1$ then $y \sim z$, then $x_1$ is the same as $y_1$ and $y_1$ is the same as $z_1$,
so $x_1$ is the same as $z_1$, and hence $x \sim z$, and therefore it is transitive. \newline
\newline
Finally, because we have shown that the relation is reflexive, symmetric, and transitive, by definition
this is an equivalence relation.
\end{proof}


(iv) $x \sim y$ if $x \ge y$ \newline
\begin{proof}
We will demonstrate a proof by counterexample. Let $x = 2$ and $y = 1$. In this scenario, $x \sim y$ because
$2 \ge 1$, however $y \sim x$ is false, because that would imply $1 \ge 2$, which is absurd,
and therefore a counterexample showing that the relation is not symmetric.
Because this relation is not symmetric, it is therefore not an equivalence relation.
\end{proof}

\end{myparindent}

\end{document}
