\newlength\mystoreparindent
\newenvironment{myparindent}[1]{%
\setlength{\mystoreparindent}{\the\parindent}
\setlength{\parindent}{#1}
}{%
\setlength{\parindent}{\mystoreparindent}
}

\documentclass[a4paper]{article}
\usepackage{amssymb}
\usepackage{amsthm}
\begin{document}

\begin{myparindent}{0pt}

Kyle Kloberdanz \newline
20 November 2021 \newline

\textbf{Exercise 1.1.1}:
Given functions $\sigma: A \rightarrow B$ and $\tau: B \rightarrow C$,
prove that if $\tau \circ \sigma$ is injective, then so is $\sigma$.

\begin{proof}
Let's setup a proof by contradiction, assume $\sigma$ is not injective, then by
definition there exists two distinct values $a_1$, $a_2$ $\in A$ such that 
$\sigma(a_1) = \sigma(a_2)$.
This implies that $\tau(\sigma(a_1)) = \tau(\sigma(a_2))$
which contradicts the assumption that $x \in A$, $\tau(\sigma(x))$ is an injective function.
Hence $a_1$ and $a_2$ cannot exist and $\sigma$ must be an injective function.
\end{proof}

\end{myparindent}

\end{document}
