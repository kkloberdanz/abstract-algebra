\newlength\mystoreparindent
\newenvironment{myparindent}[1]{%
\setlength{\mystoreparindent}{\the\parindent}
\setlength{\parindent}{#1}
}{%
\setlength{\parindent}{\mystoreparindent}
}

\documentclass[a4paper]{article}
\usepackage{amssymb}
\usepackage{amsthm}
\usepackage{mathtools}

\DeclarePairedDelimiterX{\infdivx}[2]{(}{)}{%
  #1\;\delimsize\|\;#2%
}
\newcommand{\infdiv}{D\infdivx}
\DeclarePairedDelimiter{\norm}{\lVert}{\rVert}

\newtheorem{theorem}{Theorem}[section]
\newtheorem{corollary}{Corollary}[theorem]
\newtheorem{lemma}[theorem]{Lemma}

\begin{document}

\begin{myparindent}{0pt}

Kyle Kloberdanz \newline
20 November 2021 \newline

\textbf{Exercise 1.1.1}:
Given functions $\sigma: A \rightarrow B$ and $\tau: B \rightarrow C$,
prove that if $\tau \circ \sigma$ is injective, then so is $\sigma$.

\begin{proof}
We will demonstrate a proof by contradiction. Assume $\sigma$ is not injective, then by
definition there exists two distinct values $a_1$, $a_2$ $\in A$ such that
$\sigma(a_1) = \sigma(a_2)$.
This implies that $\tau(\sigma(a_1)) = \tau(\sigma(a_2))$
which contradicts the assumption that $x \in A$, $\tau(\sigma(x))$ is an injective function.
Hence $a_1$ and $a_2$ cannot exist and $\sigma$ must be an injective function.
\end{proof}

\textbf{Exercise 1.2.2}:
Decide which of the following are equivalence relations on the set of natural numbers $\mathbb{N}$. For those
that are, prove it. For those that are not, explain why. \newline
\newline

(i) $x \sim y$ if $\norm{x - y} \le 3$ \newline
\begin{proof}
We will demonstrate a proof by counterexample.
We will demonstrate that there exists a situation such that $x \sim y$ and $y \sim z$ but $x \sim z$ is false.
Let $x, y, z \in \mathbb{N}$ and let $x = 2$, $y = 4$, and $z = 6$. \newline
\newline
We can see that for the case of $x \sim y$ that $\norm{2 - 4} \le 3$ is true. \newline
We can see that for the case of $y \sim z$ that $\norm{4 - 6} \le 3$ is true. \newline
However, we can see that for the case of $x \sim z$ that $\norm{2 - 6} \le 3$ is false. \newline
Hence, the statement is not transitive, and therefore it is not an equivalence relation.
\end{proof}

(ii) $x \sim y$ if $\norm{x - y} \ge 3$ \newline
\begin{proof}
For any $x \in \mathbb{N}, x - x = 0$. Furthermore $\norm{x - x} = 0$. Clearly 0 is less than 3,
hence the statement $\norm{x - x} \ge 3$ is false,
hence $x \sim x$ is false, which means $x \sim y$ is not reflexive and therefore not an equivalence relation. \newline
\end{proof}

(iii) $x \sim y$ if $x$ and $y$ have the same digit in the 1's place (expressed in base 10). \newline
\begin{proof}
Let the set $D$ be the set of decimal digits, which would be exactly $\{0, 1, 2, 3, 4, 5, 6, 7, 8, 9\}$.
Note that $D \subset \mathbb{N}$.
Let $x_1, y_1, z_1 \in D$ be the digit in the one's place of $x, y, z \in \mathbb{N}$ respectively.
which means that a statement such as $x_1 = y_1$ would be read as $x$ and $y$ have the
same digit in the one's place.\newline
\newline

First, for any natural number $n \in \mathbb{N}, n$ is the same as itself,
hence $x_1 = x_1$ and $y_1 = y_1$, and therefore $x \sim x$ and $y \sim y$ so it is reflexive. \newline
\newline

Second, observe that if $x \sim y$, then $x_1 = y_1$, furthermore $x_1$ is the same number as $y_1$, but then $y_1$ is the same number as $x_1$,
and hence $y \sim x$, so it is symmetric. \newline
\newline
Third, if $x \sim y$ then $x_1 = y_1$ and if $y \sim z$ then $y_1 = z_1$, then $x_1$ is the same as $y_1$ and $y_1$ is the same as $z_1$,
so $x_1$ is the same as $z_1$, and hence $x \sim z$, and therefore it is transitive. \newline
\newline
Finally, because we have shown that the relation is reflexive, symmetric, and transitive, by definition
this is an equivalence relation.
\end{proof}


(iv) $x \sim y$ if $x \ge y$ \newline
\begin{proof}
We will demonstrate a proof by counterexample. Let $x = 2$ and $y = 1$. In this scenario, $x \sim y$ because
$2 \ge 1$, however $y \sim x$ is false, because that would imply $1 \ge 2$, which is absurd,
and therefore a counterexample showing that the relation is not symmetric.
Because this relation is not symmetric, it is therefore not an equivalence relation.
\end{proof}

\textbf{Exercise 1.2.3}:
Prove that for any $f: X \rightarrow Y$, the relation $\sim_f$ defined in Example 1.2.6 is an equivalence relation.
Show that for any $x \in X$, the equivalence class of $x$ is precisely $[x] = f^{-1}(f(x))$. \newline
\newline
\textit{Note}: $f^{-1}$ refers to the \textit{preimage} of \textit{fiber} of $f$, which is defined on page 4 of your textbook.
You should \textit{not} assume that $f$ is an invertible function.
\newline

Recall that Example 1.2.6 states: Suppose $f : X \rightarrow Y$ is a function.
Define a relation $\sim_f$ on $X$ by $x \sim_f y$ if $f(x) = f(y)$.

\begin{proof}
    First, we will show that the relation is an equivalence relation. \newline

    Because $f(x)$ is the same as $f(x)$, we can see that $f(x) = f(x)$,
    so therefore we can see that the relation is reflexive. \newline

    If $f(x) = f(y)$, then $f(x)$ is the same as $f(y)$, so $f(y) = f(x)$,
    and therefore symmetric. \newline

    If $f(x) = f(y)$ and $f(y) = f(z)$ where $z \in X$, then $f(x)$ is
    the same as $f(y)$, but then $f(y)$ is the same as $f(z)$, hence $f(x) = f(z)$
    therefore the relation is transitive. \newline

    Because the relation is reflexive, symmetric, and transitive, it
    is therefore an equivalence relation. \newline

    Next, we will show that for any $x \in X$, the equivalence class of x is
    precisely $[x] = f^{-1}(f(x))$. \newline

    Since $[x]$ is the set $\{y \in X | y \sim_f x\}$, which when substituting
    the definition of $\sim_f$ is $\{y \in X | f(y) = f(x)\}$. With the preimage
    of $f(x)$ being defined as $f^{-1}(f(x)) = \{ y \in X | f(y) = f(x) \}$.
    We can see that these sets are precisely the same, so therefore $[x]$ is
    precisely $f^{-1}(f(x))$.
\end{proof}

\textbf{Exercise 1.2.4}:
Prove Proposition 1.2.13: If $\sim$ is an equivalence relation on $X$, then the function $\pi: X \rightarrow X/\sim$
defined by $\pi(x) = [x]$ is a surjective map, and the equivalence relation $\sim_{\pi}$ determined by $\pi$ is
precisely $\sim$. \newline

\begin{proof}
TODO \newline
\end{proof}

\textbf{Exercise 1.3.1}:
Let $\sigma \in S_8$ be the permutation given by the $2 \times 8$ matrix \newline
\[
  \sigma =
  \begin{pmatrix}
    1 & 2 & 3 & 4 & 5 & 6 & 7 & 8 \\
    4 & 1 & 3 & 2 & 7 & 6 & 8 & 5 \\
  \end{pmatrix}
\]
Express $\sigma, \sigma^2, \sigma^3$ and $\sigma^{-1}$ in disjoint cycle notation. \newline

For $\sigma$:

$\sigma(1) = 4$ \newline
$\sigma(2) = 1$ \newline
$\sigma(3) = 3$ \newline
$\sigma(4) = 2$ \newline
$\sigma(5) = 7$ \newline
$\sigma(6) = 6$ \newline
$\sigma(7) = 8$ \newline
$\sigma(8) = 5$ \newline

In disjoint cycle notation:
\[
    \noindent \sigma = \begin{array}{rrr} (1 & 4 & 2) \end{array}
    \noindent \begin{array}{r} (3) \end{array}
    \noindent \begin{array}{r} (6) \end{array}
    \noindent \begin{array}{rrr} (5 & 7 & 8) \end{array}
\]

For $\sigma^2$:

$\sigma(\sigma(1)) = \sigma(4) = 2$ \newline
$\sigma(\sigma(2)) = \sigma(1) = 4$ \newline
$\sigma(\sigma(3)) = \sigma(3) = 3$ \newline
$\sigma(\sigma(4)) = \sigma(2) = 1$ \newline
$\sigma(\sigma(5)) = \sigma(7) = 8$ \newline
$\sigma(\sigma(6)) = \sigma(6) = 6$ \newline
$\sigma(\sigma(7)) = \sigma(8) = 5$ \newline
$\sigma(\sigma(8)) = \sigma(5) = 7$ \newline

\[
  \sigma^2 =
  \begin{pmatrix}
    1 & 2 & 3 & 4 & 5 & 6 & 7 & 8 \\
    2 & 4 & 3 & 1 & 8 & 6 & 5 & 7 \\
  \end{pmatrix}
\]

In disjoint cycle notation:
\[
    \noindent \sigma^2 = \begin{array}{rrr} (1 & 2 & 4) \end{array}
    \noindent \begin{array}{r} (3) \end{array}
    \noindent \begin{array}{r} (6) \end{array}
    \noindent \begin{array}{rrr} (5 & 8 & 7) \end{array}
\]

For $\sigma^3$:

$\sigma(\sigma(\sigma(1))) = \sigma(\sigma(4)) = \sigma(2) = 1$ \newline
$\sigma(\sigma(\sigma(2))) = \sigma(\sigma(1)) = \sigma(4) = 2$ \newline
$\sigma(\sigma(\sigma(3))) = \sigma(\sigma(3)) = \sigma(3) = 3$ \newline
$\sigma(\sigma(\sigma(4))) = \sigma(\sigma(2)) = \sigma(1) = 4$ \newline
$\sigma(\sigma(\sigma(5))) = \sigma(\sigma(7)) = \sigma(8) = 5$ \newline
$\sigma(\sigma(\sigma(6))) = \sigma(\sigma(6)) = \sigma(6) = 6$ \newline
$\sigma(\sigma(\sigma(7))) = \sigma(\sigma(8)) = \sigma(5) = 7$ \newline
$\sigma(\sigma(\sigma(8))) = \sigma(\sigma(5)) = \sigma(7) = 8$ \newline

\[
  \sigma^3 =
  \begin{pmatrix}
      1 & 2 & 3 & 4 & 5 & 6 & 7 & 8 \\
      1 & 2 & 3 & 4 & 5 & 6 & 7 & 8 \\
  \end{pmatrix}
\]

In disjoint cycle notation:
\[
    \noindent \sigma^3 =
    \noindent \begin{array}{r} (1) \end{array}
    \noindent \begin{array}{r} (2) \end{array}
    \noindent \begin{array}{r} (3) \end{array}
    \noindent \begin{array}{r} (4) \end{array}
    \noindent \begin{array}{r} (5) \end{array}
    \noindent \begin{array}{r} (6) \end{array}
    \noindent \begin{array}{r} (7) \end{array}
    \noindent \begin{array}{r} (8) \end{array}
\]

For $\sigma^{-1}$, we must find the inverse of $\sigma$, such that $\sigma^{-1} \circ \sigma = \sigma \circ \sigma^{-1}$ is the identity.

\[
    \sigma^{-1} =
  \begin{pmatrix}
      1 & 2 & 3 & 4 & 5 & 6 & 7 & 8 \\
      2 & 4 & 3 & 1 & 8 & 6 & 5 & 7 \\
  \end{pmatrix}
\]

Let's verify that this is correct by showing that $\sigma \circ \sigma^{-1}$ is the identity.

$\sigma^{-1}(1) = 2$ \newline
$\sigma^{-1}(2) = 4$ \newline
$\sigma^{-1}(3) = 3$ \newline
$\sigma^{-1}(4) = 1$ \newline
$\sigma^{-1}(5) = 8$ \newline
$\sigma^{-1}(6) = 6$ \newline
$\sigma^{-1}(7) = 5$ \newline
$\sigma^{-1}(8) = 7$ \newline

For $\sigma \circ \sigma^{-1}$

$\sigma(\sigma^{-1}(1)) = \sigma(2) = 1$ \newline
$\sigma(\sigma^{-1}(2)) = \sigma(4) = 2$ \newline
$\sigma(\sigma^{-1}(3)) = \sigma(3) = 3$ \newline
$\sigma(\sigma^{-1}(4)) = \sigma(1) = 4$ \newline
$\sigma(\sigma^{-1}(5)) = \sigma(8) = 5$ \newline
$\sigma(\sigma^{-1}(6)) = \sigma(6) = 6$ \newline
$\sigma(\sigma^{-1}(7)) = \sigma(5) = 7$ \newline
$\sigma(\sigma^{-1}(8)) = \sigma(7) = 8$ \newline

\[
    \sigma \circ \sigma^{-1} =
    \begin{pmatrix}
        1 & 2 & 3 & 4 & 5 & 6 & 7 & 8 \\
        1 & 2 & 3 & 4 & 5 & 6 & 7 & 8 \\
    \end{pmatrix}
\]

We can see that we have correctly found $\sigma^{-1}$ in disjoint cycle notation to be:

\[
    \noindent \sigma^{-1} = \begin{array}{rrr} (1 & 2 & 4) \end{array}
    \noindent \begin{array}{r} (3) \end{array}
    \noindent \begin{array}{r} (6) \end{array}
    \noindent \begin{array}{rrr} (5 & 8 & 7) \end{array}
\]

\textbf{Exercise 1.3.2}:
\noindent Consider $\sigma = \begin{array}{rrr} (3 & 4 & 8) \end{array}$  $\begin{array}{rrrr} (5 & 7 & 6 & 9) \end{array}$
and $\tau = \begin{array}{rrrr} (1 & 9 & 3 & 5) \end{array}$ $\begin{array}{rrrr} (2 & 7 & 4) \end{array}$
in $S_9$ expressed in disjoint cycle notation.
Compute $\sigma \circ \tau$ and $\tau \circ \sigma$, expressing both in disjoint cycle notation. \newline

In matrix form:

\[
    \sigma =
    \begin{pmatrix}
        1 & 2 & 3 & 4 & 5 & 6 & 7 & 8 & 9 \\
        1 & 2 & 4 & 8 & 7 & 9 & 6 & 3 & 5 \\
    \end{pmatrix}
\]

\[
    \tau =
    \begin{pmatrix}
        1 & 2 & 3 & 4 & 5 & 6 & 7 & 8 & 9 \\
        9 & 7 & 5 & 2 & 1 & 6 & 4 & 8 & 3 \\
    \end{pmatrix}
\]

For $\sigma$:

$\sigma(1) = 1$ \newline
$\sigma(2) = 2$ \newline
$\sigma(3) = 4$ \newline
$\sigma(4) = 8$ \newline
$\sigma(5) = 7$ \newline
$\sigma(6) = 9$ \newline
$\sigma(7) = 6$ \newline
$\sigma(8) = 3$ \newline
$\sigma(9) = 5$ \newline

For $\tau$:

$\tau(1) = 9$ \newline
$\tau(2) = 7$ \newline
$\tau(3) = 5$ \newline
$\tau(4) = 2$ \newline
$\tau(5) = 1$ \newline
$\tau(6) = 6$ \newline
$\tau(7) = 4$ \newline
$\tau(8) = 8$ \newline
$\tau(9) = 3$ \newline

For $\sigma \circ \tau$:

$\sigma(\tau(1)) = \sigma(9) = 5$ \newline
$\sigma(\tau(2)) = \sigma(7) = 6$ \newline
$\sigma(\tau(3)) = \sigma(5) = 7$ \newline
$\sigma(\tau(4)) = \sigma(2) = 2$ \newline
$\sigma(\tau(5)) = \sigma(1) = 1$ \newline
$\sigma(\tau(6)) = \sigma(6) = 9$ \newline
$\sigma(\tau(7)) = \sigma(4) = 8$ \newline
$\sigma(\tau(8)) = \sigma(8) = 3$ \newline
$\sigma(\tau(9)) = \sigma(3) = 4$ \newline

In matrix form:

\[
    \sigma \circ \tau =
    \begin{pmatrix}
        1 & 2 & 3 & 4 & 5 & 6 & 7 & 8 & 9 \\
        5 & 6 & 7 & 2 & 1 & 9 & 8 & 3 & 4 \\
    \end{pmatrix}
\]

In disjoint cycle notation:

\[
    \noindent \sigma \circ \tau = \begin{array}{rr} (1 & 5) \end{array}
    \noindent \begin{array}{rrrrr} (2 & 6 & 9 & 4) \end{array}
    \noindent \begin{array}{rrr} (3 & 7 & 8) \end{array}
\]


For $\tau \circ \sigma$:

$\tau(\sigma(1)) = \tau(1) = 9$ \newline
$\tau(\sigma(2)) = \tau(2) = 7$ \newline
$\tau(\sigma(3)) = \tau(4) = 2$ \newline
$\tau(\sigma(4)) = \tau(8) = 8$ \newline
$\tau(\sigma(5)) = \tau(7) = 4$ \newline
$\tau(\sigma(6)) = \tau(9) = 3$ \newline
$\tau(\sigma(7)) = \tau(6) = 6$ \newline
$\tau(\sigma(8)) = \tau(3) = 5$ \newline
$\tau(\sigma(9)) = \tau(5) = 1$ \newline

\[
    \tau \circ \sigma =
    \begin{pmatrix}
        1 & 2 & 3 & 4 & 5 & 6 & 7 & 8 & 9 \\
        9 & 7 & 2 & 8 & 4 & 3 & 6 & 5 & 1 \\
    \end{pmatrix}
\]

In disjoint cycle notation:

\[
    \noindent \tau \circ \sigma = \begin{array}{rr} (1 & 9) \end{array}
    \noindent \begin{array}{rrrrr} (2 & 7 & 6 & 3) \end{array}
    \noindent \begin{array}{rrr} (4 & 8 & 5) \end{array}
\]

\end{myparindent}
\end{document}
