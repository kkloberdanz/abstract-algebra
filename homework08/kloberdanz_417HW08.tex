\newlength\mystoreparindent
\newenvironment{myparindent}[1]{%
\setlength{\mystoreparindent}{\the\parindent}
\setlength{\parindent}{#1}
}{%
\setlength{\parindent}{\mystoreparindent}
}

\documentclass[a4paper]{article}
\usepackage{amssymb}
\usepackage{amsthm}
\usepackage{mathtools}
\usepackage{enumitem}
\usepackage{graphicx}
\usepackage{mathrsfs}
\usepackage{amsmath}

\DeclarePairedDelimiterX{\infdivx}[2]{(}{)}{%
  #1\;\delimsize\|\;#2%
}
\newcommand{\infdiv}{D\infdivx}
\DeclarePairedDelimiter{\norm}{\lVert}{\rVert}

\newtheorem{theorem}{Theorem}[section]
\newtheorem{corollary}{Corollary}[theorem]
\newtheorem{lemma}[theorem]{Lemma}

\graphicspath{ {./} }

\begin{document}

\begin{myparindent}{0pt}

Kyle Kloberdanz \newline
Math 417 \newline

\textbf{Exercise 3.4.1}:
  Let $\tau \in S_N$ and suppose that $\sigma = (k_1 ~k_2 ~... ~k_j)$ is a
  $j$-cycle. Prove that the conjugate of $\sigma$ by $\tau$ is also a $j$-cycle,
  and is given by

  \[ \tau \sigma \tau^{-1} = (\tau(k_1) ~\tau(k_2) ~... ~\tau(k_j)) \]

  Further prove that if $\sigma' \in S_n$ is any other $j$-cycle, then $\sigma$
  and $\sigma'$ are conjugate. \textit{Hint: For the second part, you should
  explicitly find a conjugating element $\tau \in S_n$}.
\begin{proof}
\end{proof}

The \textbf{cycle structure} of an element $\sigma \in S_n$ denotes the number
of cycles of each length in the disjoint cycle representation of $\sigma$.
We can encode the cycle structure with a \textbf{partition of $n$:}
\[ n = j_1 + j_2 + ... + j_r \]

Where $\{ j_r \}$ are positive integers giving the length of the distinct cycles
(where we include "1's" for every number that is fixed, which we view as a
"1-cycle", i.e., the identity). For example, the cycle structure of
$(1 ~2 ~3)(5 ~6) \in S_6$ is $1 + 2 + 3 = 6$, since there is a 1-cycle, a
2-cycle, and a 3-cycle in the disjoint cycle representation.
\newline

\textbf{Exercise 3.4.2}:
Suppose $\sigma_1, \sigma_2 \in S_n$. Using the previous exercies, prove that
$\sigma_1$ and $\sigma_2$ have the same cycle structure if and only if they
are conjugate.
\begin{proof}
\end{proof}

\textbf{Exercise 3.4.3}:
Proposition 1.3.9 shows that every permutation is a composition of 2-cycles, and
thus the set of all 2-cycles generates $S_n$ (i.e. the subgroup $G < S_n$
generated by the set of all 2-cycles is all of $S_n$). Prove that $(1 ~2)$ and
$(1 ~2 ~3 ~... ~n) \in H$ \textit{and then} $\sigma^{k}(1 ~2)\sigma^{-k} \in H$
\textit{for} $k \ge 1$. \textit{See also Exercise 3.4.1}.
\begin{proof}
\end{proof}

\textbf{Exercise 3.4.3}:
Prove $D_3 \cong S_3$.
\begin{proof}
\end{proof}

\textbf{Exercise 3.4.8}:
Let $n \ge 3$. Prove that $R_n = \{ I, r, r^2, r^3, ..., r^{n - 1} \} \in D_n$,
the cyclic subgroup generated by $r$, is a normal subgroup. This is called the
\textbf{subgroup of rotations.}
\begin{proof}
\end{proof}

\textbf{Exercise 3.5.1}:
Prove \textit{Fermat's Little Theorem:} For every prime $p \ge 2$ and
$a \in \mathbb{Z}$, we have $a^p \equiv a \pmod p$. \textit{Hint: Consider the
two cases $p|a$ and $p \nmid a$, in the latter case thinking about the group
$\mathbb{Z}_p^{\times}$}.
\begin{proof}
  If $p | a$ then

  \[ a^p \equiv a \pmod p \implies 0 \equiv 0 \pmod p \]

  Otherwise, if $p \nmid a$, then dividing both sides by $a$, we get:

  \[ a^p \equiv a \pmod p \implies a^{p-1} \equiv 1 \pmod p \]

  Because $p$ is prime and $p \nmid a$ then there exists no common factors
  between $a$ and $p$ other than 1, thus $a^p \equiv a \pmod p$.
\end{proof}

\textbf{Exercise 3.5.4}:
Suppose $G$ is a group and $N < G$ is a subgroup with $[G ~: ~N] = 2$. Prove
that $N \triangleleft G$ is a normal subgroup.
\begin{proof}
  Recall Proposition 3.5.10, $N$ is normal in $G$ $\iff$ for all $g \in G, gN = Ng$.
  \newline
  $[G ~: N] = 2$ implies the number of left cosets of $N$ in $G$ is 2. Because
  of this, we will only need to prove this for two cases, i.e., the two cosets.
  \newline

  Let $g, h \in G$

  First coset, assume $g \in N$, then by example 3.3.10, $gN = N = Ng \implies gN = Ng$.

  Second coset, assume $h \notin N$, then $hN \neq N$. At the same time, Because
  there are only two cosets and $h \notin N$, then $h$ cannot be in the first
  coset, and therefore it can only be in the second coset, which would mean in
  this case that the left coset is equal to the right coset, hence $hN = Nh$. \newline

  Because we can show that for each coset that there exists some $g \in G$ in
  each coset such that $gN = Ng$, then by Proposition 3.5.10 $N$ is a normal
  subgroup of $G$, i.e., $N \triangleleft G$.
\end{proof}

\textbf{Exercise 3.5.6}:
Suppose $K, H < G$ are subgroups of a group $G$. Prove that for all $g \in G,
H \cap gK$ is either empty, or is equal to a coset of $K \cap H$ in $H$. Using
this, prove that

\[ [H ~: ~K \cap H] \le [G ~: ~K] \]

\textit{Hint:} To prove the inequality, define a function from $H/K \cap H
\rightarrow G/K$ and prove that it is well-defined and injective.
\begin{proof}
  Let $L < G$ such that $L = K \cup H$.

  We will start by showing that there is no overlap between cosets. Let's show
  this with a proof by contradiction. Let's assume there there is an element
  that is common to cosets of $L$ and $gL$ such that $g \notin L$.
  This implies that $g l_1 = g l_2$ for some $l_1, l_2 \in L$.

  \[ g l_1 = l_2 \implies \]
  \[ g l_1 l_1^{-1} = l_2 l_1^{-1} \implies \]
  \[ g = l_2 l_1^{-1} \]

  Because $l_1$ and $l_2$ are in $L$, $l_1 l_2 \in L$. But this implies that
  $g \in L$, which is a contradiction. Hence there can be no common elements
  between cosets of $L$, and therefore $H \cap gK$ will either be empty or
  equal to a coset of $K \cap H$ in $H$. \newline

  Because there is no overlap between cosets, and $K, H, < G$, and $|K|$ and
  $|H|$ are either equal to or less than $G$, and $|K \cap H|$ is either equal
  to or less than $|K|$.

  then the number of cosets of $K \cap H$ in $H$ cannot be larger than the
  number of cosets of $K$ in $G$, hence $[H ~: ~K \cap H] \le [G ~: ~K]$.
\end{proof}

\end{myparindent}
\end{document}
