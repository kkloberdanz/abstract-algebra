\newlength\mystoreparindent
\newenvironment{myparindent}[1]{%
\setlength{\mystoreparindent}{\the\parindent}
\setlength{\parindent}{#1}
}{%
\setlength{\parindent}{\mystoreparindent}
}

\documentclass[a4paper]{article}
\usepackage{amssymb}
\usepackage{amsthm}
\usepackage{mathtools}
\usepackage{enumitem}
\usepackage{graphicx}
\usepackage{mathrsfs}
\usepackage{amsmath}

\DeclarePairedDelimiterX{\infdivx}[2]{(}{)}{%
  #1\;\delimsize\|\;#2%
}
\newcommand{\infdiv}{D\infdivx}
\DeclarePairedDelimiter{\norm}{\lVert}{\rVert}

\newtheorem{theorem}{Theorem}[section]
\newtheorem{corollary}{Corollary}[theorem]
\newtheorem{lemma}[theorem]{Lemma}

\graphicspath{ {./} }

\begin{document}

\begin{myparindent}{0pt}

Kyle Kloberdanz \newline
Math 417 \newline

\textbf{Exercise 3.4.1}:
  Let $\tau \in S_N$ and suppose that $\sigma = (k_1 ~k_2 ~... ~k_j)$ is a
  $j$-cycle. Prove that the conjugate of $\sigma$ by $\tau$ is also a $j$-cycle,
  and is given by

  \[ \tau \sigma \tau^{-1} = (\tau(k_1) ~\tau(k_2) ~... ~\tau(k_j)) \]

  Further prove that if $\sigma' \in S_n$ is any other $j$-cycle, then $\sigma$
  and $\sigma'$ are conjugate. \textit{Hint: For the second part, you should
  explicitly find a conjugating element $\tau \in S_n$}.
\begin{proof}
\end{proof}

The \textbf{cycle structure} of an element $\sigma \in S_n$ denotes the number
of cycles of each length in the disjoint cycle representation of $\sigma$.
We can encode the cycle structure with a \textbf{partition of $n$:}
\[ n = j_1 + j_2 + ... + j_r \]

Where $\{ j_r \}$ are positive integers giving the length of the distinct cycles
(where we include "1's" for every number that is fixed, which we view as a
"1-cycle", i.e., the identity). For example, the cycle structure of
$(1 ~2 ~3)(5 ~6) \in S_6$ is $1 + 2 + 3 = 6$, since there is a 1-cycle, a
2-cycle, and a 3-cycle in the disjoint cycle representation.
\newline

\textbf{Exercise 3.4.2}:
Suppose $\sigma_1, \sigma_2 \in S_n$. Using the previous exercies, prove that
$\sigma_1$ and $\sigma_2$ have the same cycle structure if and only if they
are conjugate.
\begin{proof}
\end{proof}

\textbf{Exercise 3.4.3}:
Proposition 1.3.9 shows that every permutation is a composition of 2-cycles, and
thus the set of all 2-cycles generates $S_n$ (i.e. the subgroup $G < S_n$
generated by the set of all 2-cycles is all of $S_n$). Prove that $(1 ~2)$ and
$(1 ~2 ~3 ~... ~n) \in H$ \textit{and then} $\sigma^{k}(1 ~2)\sigma^{-k} \in H$
\textit{for} $k \ge 1$. \textit{See also Exercise 3.4.1}.
\begin{proof}
\end{proof}

\textbf{Exercise 3.4.3}:
Prove $D_3 \cong S_3$.
\begin{proof}
\end{proof}

\textbf{Exercise 3.4.8}:
Let $n \ge 3$. Prove that $R_n = \{ I, r, r^2, r^3, ..., r^{n - 1} \} \in D_n$,
the cyclic subgroup generated by $r$, is a normal subgroup. This is called the
\textbf{subgroup of rotations.}
\begin{proof}
\end{proof}

\textbf{Exercise 3.5.1}:
Prove \textit{Fermat's Little Theorem:} For every prime $p \ge 2$ and
$a \in \mathbb{Z}$, we have $a^p \equiv a \pmod p$. \textit{Hint: Consider the
two cases $p|a$ and $p \nmid a$, in the latter case thinking about the group
$\mathbb{Z}_p^{\times}$}.
\begin{proof}
\end{proof}

\textbf{Exercise 3.5.4}:
Suppose $G$ is a group and $N < G$ is a subgroup with $[G ~: ~N] = 2$. Prove
that $N \triangleleft G$ is a normal subgroup.
\begin{proof}
\end{proof}

\textbf{Exercise 3.5.6}:
Suppose $K, H < G$ are subgroups of a group $G$. Prove that for all $g \in G,
H \cap gK$ is either empty, or is equal to a coset of $K \cap H$ in $H$. Using
this, prove that

\[ [H ~: ~K \cap H] \le [G ~: ~K] \]

\textit{Hint:} To prove the inequality, define a function from $H/K \cap H
\rightarrow G/K$ and prove that it is well-defined and injective.
\begin{proof}
\end{proof}

\end{myparindent}
\end{document}
