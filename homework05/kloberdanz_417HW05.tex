\newlength\mystoreparindent
\newenvironment{myparindent}[1]{%
\setlength{\mystoreparindent}{\the\parindent}
\setlength{\parindent}{#1}
}{%
\setlength{\parindent}{\mystoreparindent}
}

\documentclass[a4paper]{article}
\usepackage{amssymb}
\usepackage{amsthm}
\usepackage{mathtools}
\usepackage{enumitem}
\usepackage{graphicx}
\usepackage{mathrsfs}
\usepackage{amsmath}

\DeclarePairedDelimiterX{\infdivx}[2]{(}{)}{%
  #1\;\delimsize\|\;#2%
}
\newcommand{\infdiv}{D\infdivx}
\DeclarePairedDelimiter{\norm}{\lVert}{\rVert}

\newtheorem{theorem}{Theorem}[section]
\newtheorem{corollary}{Corollary}[theorem]
\newtheorem{lemma}[theorem]{Lemma}

\graphicspath{ {./} }

\begin{document}

\begin{myparindent}{0pt}

Kyle Kloberdanz \newline
27 February 2022 \newline

\textbf{Exercise 2.6.3}:
Prove Proposition 2.6.8: If $(G, *)$ is a group, and $H \subset G$ is a
subgroup, then the group operation on $G$ restricts to an operation on $H$
making it into a group. \newline

\begin{proof}
  We must show that $H$ is a group. Recall a set $G$ to be a group, group it
  must satisfy the following conditions.

  (i) $*$ is associative

  (ii) there is an identity $e \in G$

  (iii) there exists an inverse $g^{-1} \in G$
  \newline

  Because were know that $H$ is a subgroup of $G$, we know by the definition Of
  a subgroup that for all $g \in H$, $g^{-1} \in H$, and therefore we know that
  there exists an inverse for every $g$ in $H$, satisfying (iii) \newline

  We also know from the definition of a subgroup, that for all $g, h \in H$ that
  $g * h \in H$. Because of this, and also from the definition that for every
  $h \in H$, that $h^{-1} \in H$, then if $h = g^{-1}$, then
  $g * h = g * g^{-1} = e$. Also, if $g = h^{-1}$ then $g * h = h^{-1} * h = e$.
  We can see that there exists an identity $e \in H$, satisfying (ii) \newline

  From the definition of a subgroup, we can see that for $f, g, h \in H$ that
  $f, g, h \in G$ and also for $x \in H$, $x = f * g * h$ that $x$ is also in
  $G$. Because we know $G$ is a group, and the operator on a group is
  associative, For $x \in G$, $x = (f * g) * h = f * (g * h)$, but because
  $x, f, g, h \in H$, then $x \in H$, $x = (f * g) * h = f * (g * h)$,
  satisfying (i), that $*$ is an associative operator for $H$. \newline

  Therefore, because we have satisfied i, ii, iii, we know that $H$ is a group
  with the operator $*$ restricted on it.
  \newline
\end{proof}

\textbf{Exercise 2.6.4}:
Prove that the roots of unity $C_n$, defined in Example 2.3.18 form a subgroup
of the group $S^1$ from Example 2.6.12. \newline

\textbf{Exercise 2.6.9}:
Suppose $R$  is a ring and $X$ is a nonempty set. Complete the proof that $R^X$
forms a ring by proving

(a) that the pointwise addition on $R^X$ is commutative,
\begin{proof}
  Recall, pointwise addition states $(f + g)(x) = f(x) + g(x)$ where $f, g \in R^X$
  Because for a ring $R$, $(R, +)$ forms an abelian group, and therefore addition
  is commutative. Because of this, we see that:
  \[ (f + g)(x) = f(x) + g(x) = g(x) + f(x) = (g + f)(x) \]
  Therefore pointwise addition on $R^X$ is commutative.
\end{proof}

(b) 0 is an additive identity,
\begin{proof}
  \[ (0 + f)(x) = 0x + f(x) = f(x) = f(x) + 0x = (f + 0)(x) \]
  Therefore, $0$ is an additive identity.
\end{proof}

(c) $-f$ is the additive inverse of any
$f \in R^X$ (and so $(R^X, +)$ is an abelian group)
\begin{proof}
  \[ (-f + f)(x) = -f(x) + f(x) = 0 = f(x) + (-f(x)) = (f + (-f))(x) \]
  Therefore, $-f$ is the additive inverse of any $f \in R^X$
\end{proof}

(d) multiplication
distributes over addition. If $R$ is a commutative ring, prove that $R^X$ is a
commutative ring. If $R$ has $1$, prove that the function $1(x) = 1$ is a $1$
for $R^X$. \newline
\begin{proof}
  Recall, pointwise multiplication states $(fg)(x) = f(x)g(x)$, where $f, g \in R^X$.
  \newline

  Let $a, b, c \in R^X$
  \[ (a + b)(x) = a(x) + b(x) \]
  \[ c(a + b)(x) = c(x) (a(x) + b(x)) \]
  \[ c(a + b)(x) = (c(x) a(x)) + (c(x) b(x)) \]
  \[ c(a + b)(x) = (c(x) a(x)) + (c(x) b(x)) \]
  \[ c(a + b)(x) = (ca)(x) + (cb)(x) \]
  \[ c(a + b)(x) = ((ca) + (cb))(x) \]

  Therefore, multiplication distributes over addition.
\end{proof}

\textbf{Exercise 2.6.11}:
Suppose $\mathbb{F}$ is any field. Find a pair of linear transformations
$S, T \in \mathcal{L}(\mathbb{F}^2, \mathbb{F}^2)$ such that $ST \neq TS$.
\newline

\textbf{Exercise 3.1.1}:
Prove part (ii) of Proposition 3.1.1: \newline

Note: for brevity, I'll omit $*$, and instead show $g * h$ as $gh$.

(i) If $g, h \in G$ and either $g * h = h$ or $h * g = h$, then $g = e$.
\begin{proof}
  \[ g^{-1}gh = g^{-1}h \]
  \[ eh = g^{-1}h \]
  \[ ehh^{-1} = g^{-1}hh^{-1} \]
  \[ e = g^{-1} \]
  \[ ge = gg^{-1} \]
  \[ g = e \]
\end{proof}

(ii) If $g, h \in G$ and $g * h = e$ then $g = h^{-1}$ and $h = g^{-1}$
\begin{proof}
  \[ gh = e \]
  \[ g^{-1}gh = g^{-1}e \]
  \[ h = g^{-1}e \]
  \[ h = g^{-1} \]

  At the same time:

  \[ gh = e \]
  \[ ghh^{-1} = eh^{-1} \]
  \[ g = eh^{-1} \]
  \[ g = h^{-1} \]
\end{proof}

\textbf{Exercise 3.1.3}:
Suppose that $G$ is a nonempty set with an associative operation $*$ such that
the following holds: \newline
1. There exists an element $e \in G$ so that $e * g = g$ for all $g \in G$, and

2. For all $g \in G$, there exists an element $g^{-1} \in G$ so that
$g^{-1} * g = e$.

Prove that $(G, *)$ is a group. \newline

The difference between this and the definition of a group is that we are only
assuming that $e$ is a “left identity”, and that elements have a “left inverse”.
Of course, we could have replaced “left” with “right” and there is an analogous
definition of a group. Hint: Start by proving that if $g \in G$ and
$g * g = g$, then $g = e$. Then prove that $g * g^{-1} = e$ (that is, the left
inverse is also a right inverse for the left identity). Finally, prove that the
left identity is also a right identity. \newline

\begin{proof}
  First, we will show that for $g \in G$, if $g * g = g$, then $g = e$. For
  brevity, I will omit the $*$.
  \[ gg = g \]
  \[ g^{-1}gg = g^{-1} g \]
  \[ eg = e \]
  \[ g = e \]

  Second, we will show that $e = g^{-1}g = gg^{-1}$
  \[ e = g^{-1}g = eg^{-1}g = (({g^{-1}})^{-1}g^{-1}) (g^{-1}g) = (({g^{-1}})^{-1}g) e = gg^{-1}e \]
  Because $e = gg^{-1}e$, and we know that $e$ is a left identity, then the only
  way for $e = gg^{-1}e$, is if $gg^{-1} = e$, which yields $e = ee$ (left identity for $e$), hence $gg^{-1} = g^{-1}g$,
  a right inverse. \newline

  Third, we will find the right identity. Because $g^{-1}g = gg^{-1}$,
  \[ g = eg = (gg^{-1})g = g(g^{-1}g) = ge = g \]

  Finally, because we know that $*$ is associative (given by the problem
  statement), and there exists an identity $e \in G$ such that
  $e * g = g * e = g$ for all $g \in G$, and for all $g \in G$, there exists
  an inverse $g^{-1} \in G$, such that $g * g^{-1} = g^{-1} * g = e$, we know
  that $G$ is indeed a group.
\end{proof}

\end{myparindent}
\end{document}
