\newlength\mystoreparindent
\newenvironment{myparindent}[1]{%
\setlength{\mystoreparindent}{\the\parindent}
\setlength{\parindent}{#1}
}{%
\setlength{\parindent}{\mystoreparindent}
}

\documentclass[a4paper]{article}
\usepackage{amssymb}
\usepackage{amsthm}
\usepackage{mathtools}
\usepackage{enumitem}
\usepackage{graphicx}
\usepackage{mathrsfs}
\usepackage{amsmath}

\DeclarePairedDelimiterX{\infdivx}[2]{(}{)}{%
  #1\;\delimsize\|\;#2%
}
\newcommand{\infdiv}{D\infdivx}
\DeclarePairedDelimiter{\norm}{\lVert}{\rVert}

\newtheorem{theorem}{Theorem}[section]
\newtheorem{corollary}{Corollary}[theorem]
\newtheorem{lemma}[theorem]{Lemma}

\graphicspath{ {./} }

\begin{document}

\begin{myparindent}{0pt}

Kyle Kloberdanz \newline
Math 417

\textbf{Exercise 3.1.4}:
Prove that a nonempty set $G$ with an associative operation $*$ is a group if
and only if the equations $g * x = h$ and $x * g = h$ have solutions in $G$
for all $g, h \in G$. \textit{Hint: Prove that if $e * g = g$ for some g, then
$e * h = h$ for all $h \in G$. Now appeal to Exercise 3.1.3} \newline

\textit{Note:} you should \textit{not} assume that the same $x$ solves both
equations simultaneously. In other words, prove that $G$ is a group if and only
if $*$ is an associative operation such that: \newline

for all $g, h \in G$ there exist $x, y \in G$ such that $g * x = h$ and $y * g = h$.
\newline

\begin{proof}
  Recall the definition of a group:
  \begin{enumerate}
    \item The operation * is associative.
    \item There is an identity $e \in G$ with the property that $e * g = g * e = g$
    for all $g \in G$
    \item For all $g \in G$, there exists an inverse $g^{-1} \in G$, with the
    property that $g * g^{-1} = g^{-1} * g = e$
  \end{enumerate}

  For 1, we are given in the problem statement that $*$ is associative. \newline

  For 2, we will show that the identity exists, and the identity is unique \newline

  Let's determine that there is an identity. We will demonstrate a proof by contradiction.
  We will assume that for any set with no identity that for all
  $g, h \in G$ there exist $x, y \in G$ such that $g * x = h$ and $y * g = h$.
  Let's use the set $X = \mathbb{N} \backslash \{0\} = \{1, 2, 3, ...\}$ and the
  associative operator $+$.
  We can see that if $g = 1$ and $h = 1$ where $g, h \in X$, then there is no
  element in $X$ such that $g + x = h$ where $x \in X$. To put another way
  $1 + x = 1$ if and only if $x = 0$, which is the identity, and because $0 \notin X$,
  this contradicts the problem statement. Hence the identity
  must be present in the set for the condition above to hold. \newline

  Now let's prove that the identity element is unique for the set $G$.
  We will demonstrate this by contradiction.
  Let's start by assuming that there are two identity elements, $e, e' \in G$
  such that $eg = ge = g$ and $e'g = ge' = g$. Now if $e$ is an identity, then
  $ee' = e'$ but at the same time if $e'$ is also an identity, then $ee' = e$,
  therefore $e' = ee' = e$, a contradiction that $e$ and $e'$ are different
  identities, hence the identity $e \in G$ is unique. \newline

  By showing that the identity exists and is unique in $G$, we know that if $e$ is the
  identity, then $e * g = g$ and $e * h = h$. \newline

  For 3, we will demonstrate another proof by contradiction. We will show that
  the condition in the problem statement cannot hold if there does not exist
  an inverse. We have already shown that our set must contain the identity.
  Let us examine the natural numbers, $\mathbb{N} = \{ 0, 1, 2, ... \}$.
  $g, g^{-1} \in N$ if and only if $0 = g + g^{-1}$ for all $g$. Clearly, the only
  value for $g$ that satisfies this is when $g = g^{-1} = 0$. For all other
  values of $g$, there does not exist a $g^{-1}$ such that $g + g^{-1} = 0$.
  Hence the inverse must exist in $G$ for the condition in the problem statement
  to hold. \newline

  We have shown that: \newline
  For all $g, h \in G$ there exist $x, y \in G$ such that $g * x = h$ and $y * g = h$
  $\iff$ $*$ is associative \textbf{and} there is an identity $e \in G$ with the
  property that $e * g = g * e = g$ for all $g \in G$ \textbf{and} for all
  $g \in G$, there exists an inverse $g^{-1} \in G$, with the property that
  $g * g^{-1} = g^{-1} * g = e$, which is the definition of a group, therefore
  a nonempty set $G$ with an associative operation $*$ is a group if
  and only if the equations $g * x = h$ and $x * g = h$ have solutions in $G$
  for all $g, h \in G$.

\end{proof}


\textbf{Exercise 3.2.1}:
Suppose $n \ge 2$ is an integer and $d, d' > 0$ are two divisors of $n$. Prove
that $\langle [d] \rangle < \langle [d'] \rangle$ if and only if $d'|d$.
\newline

\begin{proof}
Let's start by assuming that $\langle [d] \rangle < \langle [d'] \rangle$.
Let $x, n \in \mathbb{Z}$.

\[ \langle [d] \rangle < \langle [d'] \rangle \implies [d] \in \langle [d'] \rangle \implies [d] = [d']^x \implies [d]_n = [d']_n^x \implies \]
\[ d \equiv d'x \pmod n \implies d' | d \]

Working backwards:

\[ d' | d \implies d \equiv d'x \pmod n \implies [d]_n = [d']_n^x \implies [d] = [d']^x \implies \]
\[ [d] \in \langle [d'] \rangle \implies \langle [d] \rangle < \langle [d'] \rangle \]

Because we have shown the implication goes both ways, therefore

\[ \langle [d] \rangle < \langle [d'] \rangle \iff d'|d \]
\end{proof}

\textbf{Exercise 3.2.2}:
Prove that the number of elements of order $n$ in $\mathbb{Z}_n$ is exactly
$\varphi(n)$, the Euler phi function of $n$.

\textit{Hint: You need to decide which $[a] \in \mathbb{Z}_n$ generate $\mathbb{Z}_n$}.
\newline

\textit{Note:} The group operation in $\mathbb{Z}_n$ is $+$ (modular addition)
\newline

\begin{proof}
  Recall, an equivalency class $[a]_n = \{ a + nk | k \in \mathbb{Z} \}$
  Recall that $1 = gcd(a, n) \iff 1 = ma + nk$ where $a, m, k \in \mathbb{Z}$.
  Because the generator for the integer is $\langle 1 \rangle$, we need to have
  some $a$ where $a + nk = [a + 1]$. If $gcd(a, n) \neq 1$, then we would be in
  a situation where there is no $a$ such that $a + nk = [a + 1]$, which would
  imply that this particular $a$ could not be a generator of $\mathbb{Z}_n$.
  Hence the only generators of $\mathbb{Z}_n$ are the elements of $\mathbb{Z}_n$
  that are relatively prime with $n$. The cardinality of this set is defined as
  $\varphi(n)$, therefore the number of elements of order $n$ in $\mathbb{Z}_n$
  is exactly $\varphi(n)$, the Euler phi function of $n$.
\end{proof}

\textbf{Exercise 3.2.3}
Draw the subgroup lattice for the groups
$\mathbb{Z}_8, \mathbb{Z}_{15}, \mathbb{Z}_{24}$, and $\mathbb{Z}_{30}$.
\newline

\includegraphics{lattice-Z8-Z15}
\includegraphics{lattice-Z24}
\includegraphics{lattice-Z30}

\textbf{Exercise 3.2.4}
Draw the subgroup lattice for $S_3$ (a group with respect to composition $\circ$).
You will need to find all the subgroups $H < S_3$ by hand (because we don't yet
have any theorems that tell us what the subgroups of $S_3$ are).
\textit{Hint: There are exactly 6 subgroups, but you should verify this by proving
that there are no other subgroups than the ones you have listed.}
\newline

To start, we will find the subgroups. \newline

Recall, any subgroup will have the following properties:
\begin{enumerate}
  \item Each group will contain the identity, $e$
  \item For all $g, h \in H$, $g * h \in H$
  \item For all $g \in H$, $g^{-1} \in H$
\end{enumerate}

Hence, each of the subgroups we find for $S_3$ must also have these properties.
To put this in plain English, each group we find will include the identity, it
will include the product of every other member of the group, and it will also
include the inverse of every member of the group. \newline

Recall from section 1.3 on permutations that $S_n = \textit{Sym}({1..n})$, so we
can use this definition to write $S_3$ as $\textit{Sym}(\{1, 2, 3\})$. \newline

We can expand out these symmetries into the set

\[ S_3 = \{e, (2 ~3), (1 ~2), (1 ~2 ~3), (1 ~3 ~2), (1 ~3) \} \]

As we can see, there are six elements in this set, which if we remember back to
probability and statistics classes, this makes sense because the number of
permutations of $3$ distinct items is $3! = 3 \times 2 \times 1 = 6$ permutations. \newline

We can also employ the powerful CAS system, Sage Mathematics, to automate this task.
\begin{verbatim}
sage: G = SymmetricGroup(3)
sage: {i for i in G}
{(), (2,3), (1,2), (1,2,3), (1,3,2), (1,3)}
\end{verbatim}

We will now find each of the subgroups, $H$, of this symmetry set. \newline

Let's begin with $H_1 = \{e\}$, the set that only contains the inverse. Let's
verify the three properties listed above. Notice that for the sake of brevity,
I will exclude repeats when verifying below. For example, I show that
$e \circ e = e \in H_1$, therefore we know that this will hold for $H_1$
through $H_6$, and therefore does not need to be shown each time. \newline

\begin{enumerate}
  \item $e \in H_1$
  \item $e \circ e = e \in H_1$
  \item $e \circ e = e \in H_1$
\end{enumerate}

Let $H_2 = \{e, (2 ~3)\}$.

\begin{enumerate}
  \item $e \in H_2$
  \item
  \[ e \circ (2 ~3) = (2 ~3) \circ e = (2 ~3) \in H_2 \]
  And
  \[ (2 ~3) \circ (2 ~3) = e \in H_2 \]
  \item $(2 ~3) \circ (2 ~3) = e \in H_2$
\end{enumerate}

Let $H_3 = \{e, (1 ~2)\}$.

\begin{enumerate}
  \item $e \in H_3$
  \item
  \[ e \circ (1 ~2) = (1 ~2) \circ e = (1 ~2) \in H_3 \]
  And
  \[ (1 ~2) \circ (1 ~2) = e \in H_3 \]
  \item $(1 ~2) \circ (1 ~2) = e \in H_3$
\end{enumerate}

Let $H_4 = \{e, (1 ~3)\}$.

\begin{enumerate}
  \item $e \in H_4$
  \item
  \[ e \circ (1 ~3) = (1 ~3) \circ e = (1 ~3) \in H_4 \]
  And
  \[ (1 ~3) \circ (1 ~3) = e \in H_4 \]
  \item $(1 ~3) \circ (1 ~3) = e \in H_4$
\end{enumerate}

Let $H_5 = \{e, (1 ~2 ~3), (1 ~3 ~2) \}$.

\begin{enumerate}
  \item $e \in H_5$
  \item
  \[ e \circ (1 ~2 ~3) = (1 ~2 ~3) \circ e = (1 ~2 ~3) \in H_5 \]
  And
  \[ e \circ (1 ~3 ~2) = (1 ~3 ~2) \circ e = (1 ~3 ~2) \in H_5 \]
  And
  \[ (1 ~3 ~2) \circ (1 ~2 ~3) = e \in H_5 \]
  And
  \[ (1 ~2 ~3) \circ (1 ~3 ~2) = e \in H_5 \]
  And
  \[ (1 ~2 ~3) \circ (1 ~2 ~3) = (3 ~1 ~2) \in H_5 \]
  And
  \[ (1 ~3 ~2) \circ (1 ~3 ~2) = (1 ~2 ~3) \in H_5 \]

  \item
  \[ (1 ~3 ~2) \circ (1 ~2 ~3) = e \in H_5 \]
  And
  \[ (1 ~2 ~3) \circ (1 ~3 ~2) = e \in H_5 \]
\end{enumerate}

Now the for this beast, $H_6 = \{ e, (2 ~3), (1 ~2), (1 ~2 ~3), (1 ~3 ~2), (1 ~3) \}$
\newline

To satisfy property 1, we see that $e \in H_6$.

We arrive at $H_6$ by the procedure below.
\begin{enumerate}
  \item Start with another set, for example $H_2$. We will see later on that we
  can start with any of our $H$ sets from above, and this procedure will still
  work.
  \item Append another item from $S_3$ into $H_6$.
  \item Calculate to composition of that element by each element already in
  $H_6$ and append that element into $H_6$.
  \item Repeat step 3 until the composition of every item has been calculated
  and appended to $H_6$.
\end{enumerate}

By following the procedure above, we construct $H_6$, which interestingly enough
happens to also be $S_3$. This procedure will satisfy property 2. \newline

From the calculations for $H_1$ through $H_5$, we have already worked out the
inverses of each element of $H_6$, and we can see that for any element of $H_6$
that the inverse is also present in $H_6$. \newline

By following this procedure starting with any set $H$ as our starting set, we
will construct the same set $H_6$ each and every time. Therefore, we have shown
that there are precisely six subgroups of $S_3$, because by appending any other
element of $S_3$ to sets $H_2$ through $H_5$ and following this procedure, we
only ever construct $H_6$. \newline

Therefore, our six subgroups of $S_3$ are:

\[ H_1 = \{e\} \]
\[ H_2 = \{e, (2 ~3)\} \]
\[ H_3 = \{e, (1 ~2)\} \]
\[ H_4 = \{e, (1 ~3)\} \]
\[ H_5 = \{e, (1 ~2 ~3), (1 ~3 ~2) \} \]
\[ H_6 = \{ e, (2 ~3), (1 ~2), (1 ~2 ~3), (1 ~3 ~2), (1 ~3) \} \]

The lattice drawing can be found below:

\includegraphics{lattice-S3} \newline

\textbf{Exercise 3.2.5}
Prove that if $G$ and $H$ are groups and $K < G$, $J < H$ are subgroups, then
$K \times J \subset G \times H$ is a subgroup. Construct an example of a
subgroup of $\mathbb{Z}_2 \times \mathbb{Z}_2$ which is \textbf{not} of the form
$K \times J$ for some $K < \mathbb{Z}_2$ and $J < \mathbb{Z}_2$.

\begin{proof}
  First we will show that $K \times J \subset G \times H$. Let $k \in K$,
  $j \in J$, $g \in G$, and $h \in H$. Because $K < G$ and $J < H$, $k \in G$
  and $j \in H$ for all $k$ and $j$, hence for any $k$, $j$, $g$, and $h$,
  $(k, j) \subset (g, h)$. \newline

  Second, we must show that for all $(k_1, j_1), (k_2, j_2) \in K \times J$ that
  $(k_1, j_1)(k_2, j_2) \in K \times J$.
  \[ (k_1, j_1)(k_2, j_2) = (k_1 k_2, j_1 j_2) \]

  Because $K$ and $J$ are subgroups, $k_1 k_2 \in K$ and $j_1 j_2 \in J$, hence
  $(k_1 k_2, j_1 j_2) \in K \times J$. \newline

  Third, we must show that for all $(k, j) \in K \times J$ that
  $(k, j)^{-1} \in K \times J$.

  \[ (k, j)(k, j)^{-1} = (k, j)(k^{-1}, j^{-1}) = (kk^{-1}, jj^{-1}) = (e, e) \]

  Verification:
  \[ (e, e)(k, j) = (ke, je) = (k, j) = (ek, ej) = (k, j)(e, e) \]

  Because $K$ and $J$ are subgroups, $k^{-1} \in K$ and $j^{-1} \in J$.
  Hence $(k^{-1}, j^{-1}) \in K \times J$, which is the inverse of $K \times J$.
  \newline

  Because of these three reasons, $K \times J \subset G \times H$ is a subgroup,
  so therefore we can write:

  \[ K \times J < G \times H \]
\end{proof}

\end{myparindent}
\end{document}
