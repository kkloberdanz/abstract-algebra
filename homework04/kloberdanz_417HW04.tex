\newlength\mystoreparindent
\newenvironment{myparindent}[1]{%
\setlength{\mystoreparindent}{\the\parindent}
\setlength{\parindent}{#1}
}{%
\setlength{\parindent}{\mystoreparindent}
}

\documentclass[a4paper]{article}
\usepackage{amssymb}
\usepackage{amsthm}
\usepackage{mathtools}
\usepackage{enumitem}

\DeclarePairedDelimiterX{\infdivx}[2]{(}{)}{%
  #1\;\delimsize\|\;#2%
}
\newcommand{\infdiv}{D\infdivx}
\DeclarePairedDelimiter{\norm}{\lVert}{\rVert}

\newtheorem{theorem}{Theorem}[section]
\newtheorem{corollary}{Corollary}[theorem]
\newtheorem{lemma}[theorem]{Lemma}

\begin{document}

\begin{myparindent}{0pt}

Kyle Kloberdanz \newline
21 February 2022 \newline

\textbf{Exercise 2.3.3}:
Let $f = x^5 + 2x^4 + 2x^3 - x^2 - 2x - 2$ and $g = 4x^4 + 16$. Find $gcd(f, g)$
and express it as $uf + vg$
\newline

\textbf{Exercise 2.3.5}:
Prove that a polynomial $f \in \mathbb{F}[x]$ of degree 3 is irreducible in
$\mathbb{F}[x]$ if it does not have a root in $\mathbb{F}$.
\newline
\begin{proof}
  Recall, for a polynomial $f$ to be irreducible, one must be able to write
  $f = uv$ where either $u$ or $v$ is a nonzero constant polynomial, i.e.,
  either $deg(u) = 0$ or $deg(v) = 0$. \newline

  We will demonstrate a proof by contrapositive where we will show that a
  polynomial $f \in \mathbb{F}[x]$ of degree 3 has
  a root in $\mathbb{F}$ if $f$ is not irreducible (i.e., $f$ is reducible) in
  $\mathbb{F}[x]$
  \newline

  Because we are assuming that $f$ is reducible, then we can write $f = uv$
  where $deg(u) > 0$ and $deg(v) > 0$. Because $deg(uv) = deg(u) + deg(v)$ and
  $deg(f) = 3$ (as per the problem statement), then we have two cases, either
  $deg(u) = 1$ and $deg(v) = 2$ or $deg(u) = 2$ and $deg(v) = 1$. \newline

  Let's start by assuming that $deg(u) = 1$. This means that $u = a_0 + a_1x$
  where $a_0, a_1 \in \mathbb{F}$. Because $\mathbb{F}$ is a field, and
  $a_1 \neq 0$ (because if $a_1 = 0$ then $deg(u) = 0$, which would be a
  contradiction to the statement that $deg(u) = 1$)
  then there
  exists a multiplicative inverse for $a_1$ in $\mathbb{F}$, namely
  $a_1^{-1}$, such that $u(-a_0a_1^{-1}) = a_0 + a_1(-a_0a_1^{-1}) = a_0 + (-a_0) = 0$
  therefore $f$ has the root $-a_0a_1^{-1} \in \mathbb{F}$. \newline

  Since we have shown that a polynomial $f \in \mathbb{F}[x]$ of degree 3 has a
  root in $\mathbb{F}$ if $f$ is reducible in $\mathbb{F}[x]$, by the
  contrapositive we see that a polynomial $f \in \mathbb{F}[x]$ of degree 3 is
  irreducible in $\mathbb{F}[x]$ if it does not have a root in $\mathbb{F}$.
\end{proof}

\textbf{Exercise}:
Come up with a polynomial $g \in \mathbb{Q}[x]$ that has no roots in $\mathbb{Q}$
but is \textit{not} irreducible. \newline

let $g = x^4 + 2x^2 + 1 = (x^2 + 1)(x^2 + 1)$ \newline

We can see that $g$ is not irreducible. \newline

To avoid having to apply the quartic equation, by factoring this polynomial, we
see that all of the roots will be $x$ such that $x^2 + 1 = 0$. Hence we can
apply the quadratic equation to find the roots like so.

\[
  \frac{-0 \pm \sqrt{0^2 - 4(1)(1)}}{2(1)} = \frac{\pm \sqrt(-4)}{2} =
  \frac{\pm 2 \sqrt{-1}}{2} = \pm \sqrt{-1} = \pm i \notin \mathbb{Q}
\]

Hence, we see that the roots to $g = x^4 + 2x^2 + 1$ are $i$ and $-i$, which
do not exist in $Q$, and is therefore an example of a polynomial
$g \in \mathbb{Q}[x]$ that has no roots in $\mathbb{Q}$ but is not irreducible.
\newline

\textbf{Exercise 2.3.6}:
Consider the polynomial $f(x) = x^3 - x + 2 \in \mathbb{Z}_5[x]$ (more precisely,
$f(x) = [1]x^3 - [1]x + [2]$). Prove that $f$ is irreducible in $\mathbb{Z}_5$.
\textit{Hint: Use Exercise 2.3.5}.
\newline
\begin{proof}
  Recall that we have proven in Exercise 2.3.5 that if a polynomial
  $f \in \mathbb{F}[x]$ of degree 3 does not have a root in $\mathbb{F}$, then
  $f$ is irreducible if $\mathbb{F}[x]$. \newline

  We will therefore show that $f \in \mathbb{Z}_5[x]$ does not have a root in
  $\mathbb{Z}_5$, and is therefore irreducible. \newline

  Because $\mathbb{Z}_5$ is the finite set $\{0, 1, 2, 3, 4\}$, we can show
  that given any $x \in \mathbb{Z}_5$, $x^3 - x + 2 \neq 0$ as follows.
  \newline

  $f(0) = 2 \neq 0$ \newline
  $f(1) = 1 - 1 + 2 = 2 \neq 0$ \newline
  $f(2) = 2^3 - 2 + 2 = 8 - 2 + 2 \equiv 8 \pmod{5} = 3 \neq 0$ \newline
  $f(3) = 3^3 - 3 + 2 = 8 - 2 + 2 = 27 - 3 + 2 \equiv 26 \pmod{5} = 1 \neq 0$ \newline
  $f(4) = 4^3 - 4 + 2 = 64 - 4 + 2 \equiv 62 \pmod{5} = 2 \neq 0$ \newline

  We have exhausted every possible value for $x \in \mathbb{Z}_5$, and found
  that no such value for $x$ is a root for $f$, i.e., where $f(x) = 0$, hence
  there does not exist a root for $f$ in $\mathbb{Z}_5$, therefore by Exercise
  2.3.5, we conclude that $f$ is irreducible in $\mathbb{Z}_5$.
\end{proof}

\textbf{Exercise}:
Consider the polynomial $p(x) = 3x^3 + 2x^2 + 4x + 2 \in \mathbb{Z}_7[x]$
(more precisely, $f(x) = [3]x^3 + [2]x^2 + [4]x + [2]$). Prove that $f$ is
\textit{not} irreducible in $\mathbb{Z}_7[x]$.


\begin{proof}
  Recall that we have proven in Exercise 2.3.5 that if a polynomial
  $f \in \mathbb{F}[x]$ of degree 3 does not have a root in $\mathbb{F}$, then
  $f$ is irreducible if $\mathbb{F}[x]$. \newline

  This means that if a polynomial
  $f \in \mathbb{F}[x]$ of degree 3 does have a root in $\mathbb{F}$, then
  $f$ is not irreducible if $\mathbb{F}[x]$. \newline

  We will show that $f$ has a root in $\mathbb{F}$. \newline

  $f(0) = 2 \neq 0$ \newline
  $f(1) = 3 + 2 + 4 + 2 \equiv 11 \pmod 7 = 2 \neq 0$ \newline
  $f(2) = 24 + 8 + 8 + 2 \equiv 42 \pmod 7 = 0$ \newline

  We have found a root $2 \in \mathbb{Z}_7$ for $f \in \mathbb{Z}_7[x]$.
  Therefore, because the polynomial $f \in \mathbb{Z}_7[x]$ of degree 3 has a
  root in $\mathbb{Z}_7$, $f$ is not irreducible.
\end{proof}

As a bonus, by using the Computer Algebra System, sage mathematics,
one can find that factors for this polynomial are $(3x + 1)(x^2 + 5x + 2)$,
therefore $p(x)$ is not irreducible.

\begin{verbatim}
sage: x = PolynomialRing(RationalField(), 'x').gen()
sage: f = (3*x^3 + 2*x^2 + 4*x + 2)
sage: f.factor_mod(7)
(3) * (x + 5) * (x^2 + 5*x + 2)
\end{verbatim}

\textbf{Exercise 2.5.1}:
Suppose $T:R^n \rightarrow R^n$ is a linear transformation. Prove that $T$ is
an isometry if and only if $T(v) \cdot T(w) = v \cdot w$. Recall that an
isometry is a \textit{bijection} that preserves distance.

\textit{Note}: when
proving that if $T(v) \cdot T{w} = v \cdot w$ then $T$ is an isometry,
make sure you verify that $T$ is a bijection.
\newline
\begin{proof}
\end{proof}

\end{myparindent}
\end{document}
