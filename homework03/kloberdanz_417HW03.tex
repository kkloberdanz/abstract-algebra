\newlength\mystoreparindent
\newenvironment{myparindent}[1]{%
\setlength{\mystoreparindent}{\the\parindent}
\setlength{\parindent}{#1}
}{%
\setlength{\parindent}{\mystoreparindent}
}

\documentclass[a4paper]{article}
\usepackage{amssymb}
\usepackage{amsthm}
\usepackage{mathtools}
\usepackage{enumitem}

\DeclarePairedDelimiterX{\infdivx}[2]{(}{)}{%
  #1\;\delimsize\|\;#2%
}
\newcommand{\infdiv}{D\infdivx}
\DeclarePairedDelimiter{\norm}{\lVert}{\rVert}

\newtheorem{theorem}{Theorem}[section]
\newtheorem{corollary}{Corollary}[theorem]
\newtheorem{lemma}[theorem]{Lemma}

\begin{document}

\begin{myparindent}{0pt}

Kyle Kloberdanz \newline
24 January 2022 \newline

\textbf{Exercise 2.1.1}:
Prove that multiplication of complex numbers is associative. More precisely,
let $z = a + bi$, $w = c + di$, and $v = g + hi$, and prove that
$z(wv) = (zw)v$.

\begin{proof}
    We must show that $z(wv) = (zw)v$ \newline

    Let's start by assuming that they are equal.

    $z(wv) = (zw)v$ \newline
    \newline
    $(a + bi)((c + di)(g + hi)) = ((a + bi)(c + di))(g + hi)$ \newline
    \newline
    $(a + bi)(cg+ chi + dgi - dhi) = (ac + adi + bci - bd)(g + hi) $ \newline
    \newline
    $acg + achi + adgi - adh + bcgi - bch - bdg - bdhi = $ \newline
    $acg + adgi + bcgi - bdg + achi - adh - bch - bdhi$ \newline

    When alligned, we see that: \newline
    $acg + achi + adgi - adh + bcgi - bdg - bch - bdhi = $ \newline
    $acg + achi + adgi - adh + bcgi - bdg - bch - bdhi$ \newline

    Therefore we can see that $z(wv) = (zw)v$ \newline
\end{proof}

\textbf{Exercise 2.1.2}:
Let $z = a + bi$, $w = c + di \in \mathbb{C}$ and prove each of the following
statements:

\begin{enumerate}[label=(\roman*)]
  \item $z + \overline{z}$ is real and $z - \overline{z}$ is imaginary.
    \begin{proof}
        First let's begin with proving $z + \overline{z}$ is real. \newline
        Recall that if $z = a + bi$, then $\overline{z} = a - bi$. \newline
        Notice that complex numbers are the set $\{ a + bi | a, b \in \mathbb{R} \}$

        $z + \overline{z} = (a + bi) + (a - bi) = 2a + bi - bi = 2a$ \newline

        Because $2 \in \mathbb{R}$ and $a \in \mathbb{R}$, and real numbers are
        closed under multiplication, $2a$ is also real, therefore
        $z + \overline{z}$ is real. \newline

        Next, we will prove that $z - \overline{z}$ is imaginary. \newline
        $z - \overline{z} = (a + bi) - (a - bi) = a - a + bi + bi = 2bi$. \newline

        Because $2 \in \mathbb{R}$ and $b \in \mathbb{R}$, and real numbers
        are closed under multiplication, then $2b \in \mathbb{R}$. Imaginary
        numbers are the set $\{ ci | c \in \mathbb{R} \}$. We can see that
        $2bi \in \{ ci | c \in \mathbb{R} \}$, therefore $z - \overline{z}$
        is imaginary.
    \end{proof}

  \item $\overline{z + w} = \overline{z} + \overline{w}$.
    \begin{proof}
        $z + w = (a + bi) + (c + di) = a + bi + c + di = (a + c) + (b + d)i$ \newline

        $\overline{z + w} = (a + c)- (c + d)i = a + c - bi - di = $
        $(a - bi) + (c - di) = \overline{z} + \overline{w}$
    \end{proof}

  \item $\overline{zw} = \overline{z}$ $\overline{w}$.
    \begin{proof}
        $zw = (a + bi)(c + di) = ac + adi + bci -bd = (ac - bd) + (ad + bc)i$ \newline

        $\overline{zw} = (ac - bd) - (ad + bc)i = ac - bd - adi - bci =$ \newline
        $(a - bi)(c - di) = \overline{z}$ $\overline{w}$
    \end{proof}
\end{enumerate}

\textbf{Exercise 2.2.1}:
Prove that if $\mathbb{F}$ is a field and $a, b \in \mathbb{F}$ with $ab = 0$,
then either $a = 0$ or $b = 0$.

\begin{proof}
TODO
\end{proof}

\textbf{Exercise 2.2.2}:
Prove that $\mathbb{Q}(\sqrt{2})$ is a field. Hint: you should use the fact that
$\sqrt{2} \notin \mathbb{Q}$.

\begin{proof}
TODO
\end{proof}

\textbf{Exercise 2.3.1}:
Prove parts (iii)-(v) of Proposition 2.3.2:

Suppose $\mathbb{F}$ is a field.

\begin{enumerate}[label=(\roman*)]
  \item Addition and multiplication are commutative, associative operations on
  $\mathbb{F}[x]$ which restrict to the operations of addition and multiplication
  on $\mathbb{F} \subset \mathbb{F}[x]$.

  \item Multiplication distributes over addition: $f(g + h) = fg + gh$ for all
  $f, g, h \in \mathbb{F}[x]$

  \item $0 \in \mathbb{F}$ is an additive identity in $\mathbb{F}[x]: f + 0 = f$
  for all $f \in \mathbb{F}[x]$.
    \begin{proof}
    TODO
    \end{proof}

  \item Every $f \in \mathbb{F}[x]$ has an additive inverse given by
  $-f = (-1)f$ with $f + (-f) = 0$.
    \begin{proof}
    TODO
    \end{proof}

  \item $1 \in \mathbb{F}$ is the multiplicative identity in
  $\mathbb{F}[x]: 1f = f$ for all $f \in \mathbb{F}[x]$.
    \begin{proof}
    TODO
    \end{proof}
\end{enumerate}

\end{myparindent}
\end{document}
