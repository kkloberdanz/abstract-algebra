\newlength\mystoreparindent
\newenvironment{myparindent}[1]{%
\setlength{\mystoreparindent}{\the\parindent}
\setlength{\parindent}{#1}
}{%
\setlength{\parindent}{\mystoreparindent}
}

\documentclass[a4paper]{article}
\usepackage{amssymb}
\usepackage{amsthm}
\begin{document}

\begin{myparindent}{0pt}

Author: Kyle Kloberdanz \newline
Date: 20 November 2021 \newline

\textbf{Exercise 1.1.1}:
Given functions $\sigma: A \rightarrow B$ and $\tau: B \rightarrow C$,
prove that if $\tau \circ \sigma$ is injective, then so is $\sigma$.

\begin{proof}
If the cardinality of A is greater than the cardinality of C, i.e.
\[|A| > |C|\]
then $\tau \circ \sigma$ is \textbf{not} injective via the
pigeonhole principle, therefore it must be the case that
\[|A| \le |C|\]

Applying the same logic from B to C, we can see that $|B| \le |C|$.

Therefore
\[|A| \le |B|\]

By definition, $|A| \le |B|$ implies an injective mapping from A to B,\newline
thus $\sigma: A \rightarrow B$ is \underline{injective}.
\end{proof}

\end{myparindent}

\end{document}
